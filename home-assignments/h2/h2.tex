\input{/Users/daniel/github/config/preamble.sty}%available at github.com/danimalabares/config
\input{/Users/daniel/github/config/thms-eng.sty}%available at github.com/danimalabares/config

\setcounter{secnumdepth}{2}
\bibliographystyle{alpha}
\begin{document}

\begin{minipage}{\textwidth}
	\begin{minipage}{1\textwidth}
		Complex Surfaces \hfill Daniel González Casanova Azuela
		
		{\small Prof. Misha Verbitsky\hfill\href{https://github.com/danimalabares/sur}{github.com/danimalabares/sur}}
	\end{minipage}
\end{minipage}\vspace{.2cm}\hrule

\vspace{10pt}
{\huge Home assignment 2: Cohomology of local systems}

\begin{thing5}{Remark 2.1}\label{rk:2.1}\leavevmode
Throughout this assignment, all manifolds are assumed to be connected.
\end{thing5}

\begin{thing4}{Exercise 2.1}[W\(\lambda\) closed forms are W\(\lambda\) exact]\label{exer:2.1}\leavevmode
Let \(\lambda>0\) be a real number. Define \textit{\textbf{weight \(\lambda\) homogeneous forms}} on \(\mathbb{R}^n\setminus 0\) as differential forms \(\eta\) which satisfy \(\rho_t^*=t^\lambda \eta\), where \(\rho_t\) is a homothety map \(z \mapsto  tz\), \(t \in \mathbb{R}\). Prove that a closed weight \(\lambda\) form is a differential of a weight \(\lambda\) form, for any \(\lambda \neq 1\).
\end{thing4}

\begin{proof}[Solution by Bruno]\leavevmode
	(To be added.)
\end{proof}

\begin{thing4}{Exercise 2.2}[MN cohomology of Hopf manifold vanishes]\label{exer:2.2}\leavevmode
Let \(M=\mathbb{R}^n\setminus\{0\}\Big/(x \sim 2x\) be a Hopf manifold, \(\theta\) a closed, non-exact 1-form, \(d_\theta=d+\theta\), and \(H^*_\theta(M)\) cohomology of the complex \((\Lambda^*(M),d_\theta)\) (\textit{\textbf{Morse-Novikov cohomology}}). Prove that \(H^i_\theta(M)=0\) for all \(i\).
\end{thing4}

\textbf{Hint.} Use the previous exercise.

\begin{proof}[Solution]\leavevmode
In view of Exercise \hyperref[exer:2.1]{2.1} we are inclined to look for a ``weight \(\lambda\)" form in the class of any given closed differential form \(\alpha \in \Lambda^{*}(M)\). Then the \textit{usual} cohomology class of \(\alpha\) vanishes.

{\color{6}Misha:} the standard way to do this is to ``average" the forms using a compact Lie group. So maybe ask GPT to what this means exactly.
\end{proof}

\begin{thing4}{Exercise 2.3}[\(\theta\) exact \(\implies\) MN cohomology=dr cohomology]\label{exer:2.3}\leavevmode
Let \(\theta\) be an exact 1-form on a manifold \(M\). Construct an isomorphism between the complexes \((\Lambda^{i}(M),d_\theta)\) and \((\Lambda^{i}(M),d)\).
\end{thing4}

\begin{proof}[Solution by Bruno]\leavevmode
Using exponential map. Do you rememeber?
\end{proof}

\begin{thing4}{Exercise 2.4}[\(\theta\) closed \(\implies\) MN vanishes iff \(\theta\) exact]\label{exer:2.4}\leavevmode
Let \(\theta\) be a closed 1-form on a connected manifold. Prove that \(H^0_\theta(M)\neq  0\) if and only if \(\theta\) is exact.
\end{thing4}

\begin{proof}[Solution by Bruno.]\leavevmode
I don't remember. There is an easy way using sheaves?
\end{proof}

\begin{thing3}{Definition 2.1}\label{def:2.1}\leavevmode
	A \textit{\textbf{complex of vector spaces}} is a collection of \(\{A_i,i \in \mathbb{Z}\}\) of vector spaces equipped with the \textit{\textbf{differential}} \(d:A_i\to A_{i+1}\) such that \(d^2=0\);its \textit{\textbf{cohomology groups}} are \(\frac{\ker d}{\operatorname{img}d}\). \textit{\textbf{Morphism of complexes}} \(\phi:(A_*,d_A)\to (B_*,d_B)\) is a collection of homomorphisms \(\phi_i:A_i\to B_i\) commuting with the differentials. \textit{\textbf{The cone}} of such a morphism is a complex \(C_i:=B_i \oplus A_{i+1}\) with the differential \(d_B+(-1)^id_A+\phi:B_i \oplus A_{i+1}\to B_{i+1\oplus A_{i+2}}\).
\begin{align*}
	d_B+(-1)^id_A+\phi:B_i\oplus A_{i+1} &\longrightarrow B_{i+1}\oplus A_{i+2} \\
	(b^i,a^{i+1}) &\longmapsto (d_B b^i, (-1)^i d_A a^{i+1}{\color{2}\underbrace{\cancel{, \phi}}_{?}})
\end{align*}

\begin{thing4}{Exercise 2.5}\label{exer:2.5}\leavevmode
Let \(A_* \xrightarrow{\phi}B_*\) be a morphism of complexes of vector spaces, and \(C_*\) its cone. Construct a long exact sequence
\[\begin{tikzcd}[column sep=small]
	\cdots\arrow[r]&H^{i-1}(C_*)\arrow[r]&H^{i}(A_*)\arrow[r]&H^{i}(B_*)\arrow[r]&H^{i}(C_*)\arrow[r]&H^{i+1}(A_*)\arrow[r]&\cdots
\end{tikzcd}\]

\textbf{Note:} the indices are different in the answer.
\end{thing4}

\begin{upshot}\leavevmode
Functions on circle are just periodic functions on \(\mathbb{R}\).
\end{upshot}

\begin{proof}[Solution by Lada]\leavevmode
I realise that
\[\begin{tikzcd}0\arrow[r]&B_i\arrow[r]&B_i \oplus  A_{i+1}\arrow[r]&A_{i+1}\arrow[r]&0\end{tikzcd}\]
which by the fundamental theorem of homological algebra gives a long exact sequence
\[\begin{tikzcd}[column sep=small]
	\cdots\arrow[r]&H^{k}(B)\arrow[r]&H^{k}(B_i \oplus  A_{i+1})=H^{k}(C_i)\arrow[r]&H^{k}(A_{i+1})\arrow[r]&H^{k+1}(B_{i+1})\arrow[r]&\cdots
\end{tikzcd}\]
\end{proof}

\begin{thing4}{Exercise 2.6}[The trivial circle bundle]\label{exer:2.6}\leavevmode
Let \(X\) be a manifold, \(M:=X \times S^1\), and \(\pi:M \to S^1\) the standard projection. Denote by \(T^1_X(M)\) the bundle of vectors tangent to the fibers of \(\pi\), \(\Lambda^{1}_X(M)\) its dual, and \(\Lambda^{*}_X(M)\) the corresponding Grassmann algebra. Consider the de Rham differential \(d_X\) acting on \(\Lambda^{*}_X(M)\) fiberwise along \(X\).

\begin{enumerate}[label=(\alph*)]
\item {\color{6}(cohomology of the vectors tangent to fibers is cohomology of base \(\otimes\) smooth functions on circle)} Prove that the cohomology of \((\Lambda^{*}_X(M),d_X)\)is isomorphic to \(H^{*}(X)\otimes C^\infty S^1\).
\item Let \(\frac{d}{dt}\) be the derivative along the circle. Prove that \(\frac{d}{dt}\) commutes with \(d_X\).
\item Let \(C^\infty_0S^1\) denote the functions wich average to zero. Prove that \(\frac{d}{dt}\) is invertible on \(C^\infty\).
\item Prove that the kernel and the cokernel of \(H^{*}(X)\otimes C^\infty S^1\xrightarrow{d/dt}H^{*}(X)\otimes C^\infty S^1\) are naturally isomorphic to \(H^{*}(X)\).
\end{enumerate}
\end{thing4}

\begin{proof}[Solution]\leavevmode
\begin{enumerate}
\item 
\textbf{Solution by Misha.}  The key idea is not to use Künneth formula but to consider differential forms on the product as elements of the tensor product over smooth functions. We are interested in the following complex: it's fiberwise de Rham.

\[\begin{tikzcd}\text{what's the kernel?} \arrow[r]&C^\infty(X \times S^1)\arrow[r,"d"]&\Lambda^{1}(X)\otimes_{C^\infty X}C^\infty M\arrow[r,"d"]&\Lambda^{2}(X)\arrow[r]&0\end{tikzcd}\]
Something like "what's the sheaf such that this is a resolution", "what's the kernel of the first map"?

Looks like the correct answer is
\[\mathbb{R}_X \otimes_{C^\infty S^1} C^\infty S^1\]
so that sheaf. And that gives \(H^{*}(X)\otimes C^\infty S^1\).

	\iffalse Right so a form \(\alpha\) of the circle bundle is a linear operator on \(TM\). So what is \(TM\). It is a point and a vector, so \((p,e^{it}) \in X \times S^1\) and \((v,r)\in TX \oplus TS^1\). And \(\alpha\) will map \((v,r)\), or some tuple of vectors, to a number.

	I would like to have an expression for \(\alpha\) decomposed as a form on \(X\) and a form on \(S^1\). Well, just as \(T(M)=T(X) \oplus  T(S^1)\) we should have \(T^*(M)=T^*(X)\oplus T^*(S^1)\). And immediately we get that \(\alpha = \omega \oplus \eta\).

	So the cohomology class of \(\alpha\), if it is closed, …
\fi
\textbf{Dani attempt} 

Now I  remember: it's Künneth formula! As I recall,
	\[H^*(X \times S^1)=H^{*}(X) \otimes H^*(S^1).\]
%But what is the cohomology of the circle? It vanishes for \(i>1\), and \(H^1(S^1)=\mathbb{Z}\).

Now I realise: it's the cohomology of the vectors tangent to the fibers of the circle bundle! I think it's the kernel of the map induced by the projection \(\pi:X \times S^1 \to S^1\), namely
\[T_X(M)=\ker \pi_*=\{v \in TM: \pi_*(v)=0\in TS^1\}\]
So this will inject in the total bundle:
\[\begin{tikzcd}0\arrow[r]&T_X(M)\arrow[r]&T(X \times S^1)\arrow[r]&TS^1\arrow[r]&0\end{tikzcd}\]
Take Grassman algebra functor to get
\[\begin{tikzcd}0\arrow[r]&\Lambda^{^*}_X(M)\arrow[r]&\Lambda^{*}(M)\arrow[r]&\Lambda^{*}(S^1)\arrow[r]&0\end{tikzcd}\]
Giving the sequence
\[\begin{tikzcd}[column sep=small]
	\cdots\arrow[r]&0\arrow[r]&H^{0}_X(M)\arrow[r]&H^{0}(M)\arrow[r]&H^{0}(S^1)\arrow[r]&\leavevmode
\end{tikzcd}\]
\[\begin{tikzcd}[column sep=small]
	\leavevmode\arrow[r]&H^{1}_X(M)\arrow[r]&H^{1}(M)\arrow[r]&H^{1}(S^1)\arrow[r]&H^{2}_X(M)\arrow[r]&\cdots
\end{tikzcd}\]
To compute the cohomology of \(M\) we use Künneth formula:
\[
\begin{array}{|c|l|}
\hline
p+q & H^{p+q}(M) \\
\hline
0 & H^0(X) \otimes H^0(S^1) = H^0(X) \otimes \mathbb{R} \\
1 & H^1(X) \otimes H^0(S^1) \oplus H^0(X) \otimes H^1(S^1) \\
  & = H^1(X) \otimes \mathbb{R} \oplus H^0(X) \otimes \mathbb{R} \\
2 & H^0(X) \otimes H^2(S^1) \oplus H^1(X) \otimes H^1(S^1) \oplus H^2(X) \otimes H^0(S^1) \\
  & = H^1(X) \otimes \mathbb{R} \oplus H^2(X) \otimes \mathbb{R} \\
3 & H^3(X) \otimes \mathbb{R} \oplus H^2(X) \otimes \mathbb{R} \\
4 & H^4(X) \otimes \mathbb{R} \oplus H^3(X) \otimes \mathbb{R} \\
5 & H^5(X) \oplus \mathbb{R} \oplus H^4(X) \otimes \mathbb{R} \\
\hline
\end{array}
\]
Giving
\[\begin{tikzcd}[column sep=small]
	\cdots\arrow[r]&0\arrow[r]&H^{0}_X(M)\arrow[r]&H^{0}(X) \otimes \mathbb{R}\arrow[r]&H^{0}(S^1)\arrow[r]&\leavevmode
\end{tikzcd}\]
\[\begin{tikzcd}[column sep=small]
	\leavevmode\arrow[r]&H^{1}_X(M)\arrow[r]&H^{1}(X)\otimes \mathbb{R} \oplus  H^{0}(X)\otimes \mathbb{R}\arrow[r]&H^{1}(S^1)\arrow[r]&\cdots
\end{tikzcd}\]
\[\begin{tikzcd}[column sep=small]
	\leavevmode\arrow[r]&H^{2}_X(M)\arrow[r]&H^{1}(X)\otimes \mathbb{R} \oplus  H^{2}(X)\otimes \mathbb{R}\arrow[r]&0\arrow[r]&\cdots
\end{tikzcd}\]
\[\begin{tikzcd}[column sep=small]
	\leavevmode\arrow[r]&H^{3}_X(M)\arrow[r]&H^{3}(X)\otimes \mathbb{R} \oplus  H^{2}(X)\otimes \mathbb{R}\arrow[r]&0\arrow[r]&\cdots
\end{tikzcd}\]


So I wonder how to finish. Perhaps using some spectral sequence?

\iffalse if the sequence
\[\begin{tikzcd}0\arrow[r]&H^{0}_X(M)\arrow[r]&H^{0}(X)\arrow[r]&\mathbb{R}\end{tikzcd}\]
gives
\[H^{0}_X(M)\cong H^{0}(X)\otimes C^\infty S^1\]
{\color{2}Looks like not.}

\textbf{Note.} Looks like using Serre spectral sequence (adding the hypothesis that \(X\) is simply connected yields \(H^0_X(M)\cong H^0(X) \otimes H^0(S^1)\). So the question of how the smooth functions \(C^\infty S^1\) appear remains a mystery.
\fi
\item 
\textbf{Solution by Misha.} Compute de Rham in coordinates and notice it commutes que \(\frac{d}{dt}\).

\textbf{Dani attempt.} 

	Here we must think that \(\frac{d}{dt}\) is a derivation \(f \mapsto \frac{df}{dt}\Big|_{t}\), where the latter is actually a smooth function depending on \(t \in S^1\). 


	Then I'd like to see \(d_X\) as a derivation, but I don't know how. The following probably doesn't make sense: \(f \mapsto d_Xf\), which is a form, so to get a number I should pair it with a vector field. So actually there \textit{is} a canonical vector field on \(S^1\), say, unit tangent vector field in a fixed direction. This would give me a number, what number?

	The number is: \(d_Xf\left(\frac{d}{dt}\right) =\frac{d f}{d t}dt\left(\frac{d}{dt}\right) =\frac{df}{dt}\Big|_{t}\)

\item (ChatGPT.) First notice that for any \(f \in C^\infty S^1\) the integral of its derivative vanishes by the fundamental theorem of calculus. To construct an inverse define for \(g \in C^\infty S^1\) the primitive \(f(t)=\int_{0}^tg(y)dy\). Then \(\int_{S^1}f(t)dt\) {\color{4}vanishes because…}


\item 

\item Kernel is zero. (Why?) The operator is elliptic, and that's it (Misha). But probably there's a way to just find why it is an isomorphism.
\end{enumerate}
\end{proof}

\begin{thing4}{Exercise 2.7}[The trivial circle bundle continued]\label{exer:2.7}\leavevmode

\end{thing4}

\begin{thing4}{Exercise 2.8}[MN of trivial circle bundle vanishes for \(X\) compact and \(\theta=\pi^*dt\)]\label{exer:2.8}\leavevmode

\end{thing4}

\end{thing3}












\end{document}
