\input{/Users/daniel/github/config/preamble.sty}%available at github.com/danimalabares/config
\input{/Users/daniel/github/config/thms-eng.sty}%available at github.com/danimalabares/config

\setcounter{secnumdepth}{2}
\bibliographystyle{alpha}
\begin{document}

\begin{minipage}{\textwidth}
	\begin{minipage}{1\textwidth}
		K3 surfaces \hfill Daniel González Casanova Azuela
		
		{\small Prof. Misha Verbitsky\hfill\href{https://github.com/danimalabares/k3}{github.com/danimalabares/k3}}
	\end{minipage}
\end{minipage}\vspace{.2cm}\hrule

\vspace{10pt}
{\huge Home assignment 1: Hopf surfaces and Kodaira surfaces}

\begin{thing4}{Exercise 1.2}\label{exer:1.2}\leavevmode
Prove that a the primary Kodaira surface (defines, as in lecture 1, as a nilmanifold) has trivial canonical bundle.
\end{thing4}

\begin{thing4}{Exercise 1.3}\label{exer:1.3}\leavevmode
Construct a closed, non-degenerate (1,1)-form.
\end{thing4}

\begin{proof}[Solution]\leavevmode
	So the idea is to use, somehow, Lie algebra: \(\mathfrak{g}^*\), the dual of \(\mathsf{Lie}(S^1 \times Heis)\), which is defined as generated by \(x,y,z,t\) and  \([x,y]=z\).
\end{proof}

\begin{defn}\leavevmode
	A \textit{\textbf{classical Hopf surface}} is \(\frac{\mathbb{C}^n}{\left<\gamma\right>}\) with %\(|\gamma(z)|<\lambda|z|\) 
	\(\gamma=\lambda \operatorname{Id}\).

		%A \textit{\textbf{classical Hopf surface}} is \(\frac{\mathbb{C}^n}{\left<\gamma\right>}\) where \(\gamma\) is a contraction: a function such that for every compact subset \(K\) and neighbourhood \(U\) of some point \(x \in M\) there exists \(N>0\) such that \(\gamma^n(K) \subset U\) for all \(n > N\).
%	that puts every compact set \(K\) inside any neighbourhood \(U\) of a given point after a finite number $N$ of iterations, \(\gamma^n(K) \subset U\) for every \(n>N\).
\end{defn}

\begin{thing4}{Exercise 1.4}\label{exer:1.4}\leavevmode
Let \(H\) be a classical Hopf surface.
\begin{enumerate}[label=(\alph*)]
\item Prove that the holomorphic tangent bundle \(TH\) {\color{4}(is?)} globally generated (that is, for each \(x \in H\), the projection \(H^{0}(TH)\longrightarrow T_xH\) is surjective).
 \item Prove that \(H^{0}(T^*H)=0\).
\item Prove that \(H^{0}(\operatorname{Sym}^k T^* H)=0\).
\item (*) Prove that \(H^{0}(T^*H)=0\) for any Hopf linear surface.
\end{enumerate}
\end{thing4}

\begin{proof}[Solution]\leavevmode
\begin{enumerate}[label=(\alph*)]
\item Right so let's think for a bit: that the projection is surjective means that for every vector in the tangent space at any point of \(H\) there is a section that gives that vector in that point. {\color{6}Why is this called ``globally generated"? I think this means that there is a set of sections that are a basis of the vector space (=fiber) at every point. So since this is a line bundle the basis at every point is a single vector, so this just means ``having a nowhere-vanishing section"}. But guess what-- it's not a line bundle, it's the (holomorphic) tangent bundle of \(H\)! So it's generated by two vectors at every point, so we \textit{are looking for two sections --vector fields! at last!-- that are a global basis}.

	But it's not immediate to me why the two definitions are equivalent. In one direction it's obvious: if it is globally generated, then every vector at every point is a linear combination of those generating sections at that point. Right so in the other direction every vector at every point is the projection of a section at that point, that is, the \textit{value} of the section at that point. Right so this ``projection" can be understood more simply as the ``valuation function". $\mathsf{OK}$ that's just some names, and they really don't solve the problem: that there is a section valued any vector at any point doesn't give a set of sections that are a basis of the fiber at every point.

Anyway, the task is really to prove either of these conditions, so perhaps try the second one since it seems easier. So just choose a vector at some point, say \((p,v) \in TH\) and show that there a section \(s\) s.t. \(s(p)=v\).

{\color{2}…}
\end{enumerate}
\end{proof}

\begin{proof}[Solution]\leavevmode
I guess I just don't know the definition of  \(H^{0}(X)\). Looks like there's some group action involved.
\begin{enumerate}[label=(\alph*)]
\item  Homogeneous vector space then all holomorphic vector fields are globally generated. 
\item {\color{6}Recall: we are trying to show that an invariant 1-form vanishes.} Then you would pair this form with a vector field that 
\item If a vector bundle is globally generated, then more or less so is its symmetric power. Then we do the same game of taking dual.
\item A linear contraction is when has eigenvalues of norm smaller than 1. (Unsolved in class.)
\end{enumerate}
\end{proof}


\begin{thing4}{Exercise 1.6}\label{exer:1.6}\leavevmode
Let $C$ be a smooth complex curve on a non-projective complex surface \(M\). Assume that \(C \subset M\) admits a positive-dominsional family of deformations. Prove that the $C$ is a genus 1 curve.
\end{thing4}

\begin{proof}[Solution by Yulia]\leavevmode
It's an application of Adjunction Formula.

\begin{remark}\leavevmode
Kodaira theorem (lecture 1). For any non-algebraic surface
\end{remark}
\begin{thm}[Kodaira]\leavevmode
For any non-algebraic surface \(X\), if \(a,b \in \operatorname{NS}(X)= H^{1,1}(X)\cap H^{2}(X,\mathbb{Z})\), with \(a^2=0\), then \(ab=0\).
\end{thm}

Since it deforms  then \(C^2=0\) bid (can't be negative) so \(g(C)=g\). So adjunction says
 \[2g-2=K_X\cdot C+ C^2=0.\]

\begin{proof}[Proof of Adjunction formula by Yulia]\leavevmode
{\color{6}I think it's just something like tensor product of graded algebras like Künneth. ``Determinant takes direct sum to product"} \(C \subset X\).

\[\det T_{X|_{C}}=\det T_C \otimes \det (N_C)\]
but since $C$ is a curve it's al 1 dimensional so we could have not put the det.

This gives \(\Omega^{2}(\Omega_{X|_{C}})=\Lambda^{1}(C) \otimes N^*_C\)
``cotangent times conormal budndles". Right but how to get from there to adjunction formula?

Remember that adjunction formula is about relating ambient manifold to submanifold.

(\textit{\textbf{degree}} of a bundle has to do with taking a section…)

So… the previous equation translates to
\[C^2 + K_X \cdot C = K_C.\]





\[\begin{tikzcd}0\arrow[r]0&T_C\arrow[r]&T_{X|_{C}}\\arrow[r]&N_C\arrow[r]&0\end{tikzcd}\]
\end{proof}
and dualize:
\[\begin{tikzcd}0\arrow[r]&N_C^*\arrow[r]&\Omega^{1}(X|_{C})\arrow[r]&T^*_C\arrow[r]&0\end{tikzcd}\]
\end{proof}

\bibliography{bib.bib}
\end{document}
