\input{/Users/daniel/github/config/preamble.sty}%available at github.com/danimalabares/config
\input{/Users/daniel/github/config/thms-eng.sty}%available at github.com/danimalabares/config

\setcounter{secnumdepth}{2}
\bibliographystyle{alpha}
\begin{document}

\begin{minipage}{\textwidth}
	\begin{minipage}{1\textwidth}
		Complex Surfaces \hfill Daniel González Casanova Azuela
		
		{\small Prof. Misha Verbitsky\hfill\href{https://github.com/danimalabares/sur}{github.com/danimalabares/sur}}
	\end{minipage}
\end{minipage}\vspace{.2cm}\hrule

\vspace{10pt}
{\huge Home assignment 4: Fubini-Study form and its potential}

\begin{thing4}{Exercise 4.2}\label{exer:4.2}\leavevmode

\end{thing4}

\begin{proof}\leavevmode
	You can just apply chain rule to obtain the first formula. Then substitute in coordinates. You will find the following form in there:

	\[A=\operatorname{Id}-\frac{\left<v,\cdot \right>}{\|v\|^2}v\]
	so for orthogonal vectors to \(v\) you get the degeneracy of the form. So Bruno has written the following
	\[\partial \bar\partial \operatorname{log} \ell=\frac{1}{\ell^2}\sum [\ell\underbrace{ \delta_{ij}-z_i - \bar{z}_j ]}_{A_{ij}}dz_i \wedge d\bar{z}_j\]
	
	
\end{proof}

\begin{thing4}{Exercise 4.3}\label{exer:4.3}\leavevmode
\begin{enumerate}[label=(\alph*)]
\item Consider the function \(|z|^2=z\bar{z}\) on \(\mathbb{C}^*\), and let \(\rho=z \frac{d}{dz}\), where \(z\) is the complex coordinate on \(\mathbb{C}\). Prove that \(\mathsf{Lie}_\rho |z|^2=|z|^2\).
\item Prove that \(\mathsf{Lie}_\rho(\operatorname{log} |z|^2)=\mathsf{const}\).
\end{enumerate}
\end{thing4}

\begin{proof}[Solution]\leavevmode
\begin{enumerate}[label=(\alph*)]
\item
	 \begin{align*}
	\mathsf{Lie}_\rho |z|^2&=i_\rho d |z|^2 +\cancelto{0}{di_\rho |z|^2}\\
	&=i_\rho d(z \bar{z})=i_\rho \Big(\bar{z}dz +z \cancelto{0}{d \bar{z}} \Big)\\
	&=i_\rho( \bar{z}dz=\bar{z}dz\left(z \frac{d}{dz}\right) =\bar{z}z.
	\end{align*}

\item We have two functions \(f:\mathbb{R}^+ \to \mathbb{R}\) and \(g:\mathbb{R}^2 \to \mathbb{R}\). \(f\) is \(\operatorname{log}\) and \(g=z \bar{z}=x^2+y^2\). Then by definition
\begin{align*}
d(f \circ g)&=\frac{\partial }{\partial x}(f \circ g)dx+ \frac{\partial }{\partial y}(f \circ g) dy\\
&=\frac{df}{dt}\Big|_{g(x,y)}\frac{\partial g}{\partial x}dx+\frac{df}{dt}\Big|_{g(x,y)}\frac{\partial g}{\partial y}dy\\
&=\operatorname{log}'(x^2+y^2)\frac{\partial }{\partial x}(x^2+y^2)dx+\operatorname{log}'(x^2+y^2)\frac{\partial }{\partial y}(x^2+y^2)dy\\
&=\frac{1}{x^2+y^2}2xdx+\frac{1}{x^2+y^2}2ydy\\
&=\frac{2}{x^2+y^2}(xdx+ydy)
\end{align*}
$\mathsf{OK}$ now go back to your basic complex geometry knowledge to tell
\[dz=dx+\sqrt{-1}dy,\qquad d\bar{z}=dx-\sqrt{-1}dy\]
and
\[\frac{\partial }{\partial z}=\frac{1}{2}\left(\frac{\partial }{\partial x}-s^{-1}\frac{\partial }{\partial y}\right) ,\qquad \frac{\partial }{\partial \bar{z}}=\frac{1}{2}\left(\frac{\partial }{\partial x}+\sqrt{-1}\frac{\partial }{\partial y}\right) .\]
which says
\[dx=\frac{1}{2}\left(dz+d\bar{z}\right) ,\qquad dy=\frac{1}{2\sqrt{-1}}(dz-d\bar{z})\]
So that our thing of today is
\begin{align*}
\frac{2}{x^2+y^2}(xdx+ydy)&=\frac{2}{|z|^2}\Big(\operatorname{Re}z\frac{1}{2}(dz+d\bar{z})+\operatorname{Im}z\frac{1}{2\sqrt{-1}}(dz-d\bar{z})\Big)\\
			  &=\frac{1}{|z|^2}\Big(\operatorname{Re}z(dz +d\bar{z})+\frac{1}{\sqrt{-1}}\operatorname{Im}z(dz-d\bar{z})\Big)
\end{align*}
That is  \(d \operatorname{log}|z|^2\). Now evaluate in \(\rho=z\frac{d}{dz}\):
\begin{align*}
d\operatorname{log}|z|^2(\rho)&=\frac{1}{|z|^2}\Big(z\operatorname{Re}z+\frac{1}{\sqrt{-1}}z\operatorname{Im}z\Big)=\frac{z}{|z|^2}\bar{z}=1
\end{align*}


\end{enumerate}
\end{proof}

\begin{thing4}{Exercise 4}\label{exer:4}\leavevmode

\end{thing4}

\begin{proof}[Solution]\leavevmode
Misha: to look for the differential of the action; tangent vector space to the action. \(\mathbb{C}^*\) action has a real vector field. \textit{This} is this vector field.
\end{proof}

\begin{remark}[Dani]\leavevmode
The point is that the vector fields \(\rho\) and \(\bar{\rho} \) are a basis for the \(\mathbb{C}^*\)-invariant vector fields, which end up being the vector fields that survive under the quotent. So the vector fields on \(\mathbb{C}P^n\) are generated by those two. Now my question is: why is the form non-degenerate? I should find that the vector that gives degeneracy on the first problem is not vertical, i.e. is tangent to the base.
\end{remark}
\end{document}
