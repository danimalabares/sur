\input{/Users/daniel/github/config/preamble.sty}%This is available at github.com/danimalabares/config
\input{/Users/daniel/github/config/thms-eng.sty}%This is available at github.com/danimalabares/config

%\usepackage[style=authortitle-terse,backend=bibtex]{biblatex}
%\addbibresource{/Users/daniel/github/config/bibliography.bib}

\begin{document}
\bibliographystyle{alpha}

\begin{minipage}{\textwidth}
	\begin{minipage}{1\textwidth}
		 Complex Surfaces\hfill Daniel González Casanova Azuela
		
		{\small Prof. Misha Verbitsky\hfill\href{https://github.com/danimalabares/}{github.com/danimalabares/}}
	\end{minipage}
\end{minipage}\vspace{.2cm}\hrule

\vspace{10pt}
{\huge Complex surfaces}

\tableofcontents

\section{Lecture 1: Kodaira dimension and Hopf manifolds}

\subsection{Outline}
\begin{enumerate}
\item Kodaira dimension definition: the function \(P(N)=H^{0}(K^N)\) is polynomial (so probably \(h\) rathen than \(H\) right?). The degree  \(\kappa(M)\) is called \textit{\textbf{Kodaira dimension}} of \(M\). If \(P\) is identically 0, we set \(\kappa(M)=-\infty\).
\item Nilmanifolds and solvmanifolds (quotients of (solvable) Lie groups.
\item Kodaira surface definition (it's a nilmanifold).
\item Minimal models. A complex surface is \textit{\textbf{minimal}} if it does not contain a smooth rational curve with self-intersection \(-1\). Thorem by Hartshorne: for any complex surface \(M\) there exists a minimal surface \(M_1\) and a holomorphic, bimeromorphic map \(M \to M_1\). \(M_1\) is called a \textit{\textbf{minimal model for \(M\).}}
\item Kodaira theorem: a complex surface is projective iaotfh: (i) field o meromorphic functions has trascendental dimension 2, (ii)  \(M\) admits a holomorphic line bundle \(L\) with \(c_1(L)^2>0\), (iii) the Neron-Severy lattice of \(M\), \(\operatorname{NS}(M):=H^{1,1}(M)\cap H^{2}(M,\mathbb{Z})\), contains a class with positive self-intersection. 
\item Class VII and \(\operatorname{V I I}_0\) surfaces definition.
\item Hopf manifolds (Hopf manifolds are \(\operatorname{  V I I}_0\)).
\end{enumerate}

\subsection{Kodaira-Enriques classification for non-algebraic surfaces: constructions and examples}

\begin{itemize}
\item \textit{\textbf{(Primary) Kodaira surface}} can be defined as \(M:=G/\Gamma\) with the complex structure defined by the subalgebra \(\mathfrak{g}^{1,0}:=\left<x+\sqrt{-1}y,z+\sqrt{-1}t\right>\), which is actually abelian.
\end{itemize}

\subsection{Holomorphic contractions and Hopf manifolds}

\textit{\textbf{Hopf manifolds}} are quotients \(\mathbb{C}^\setminus\{0\}/ \left<\gamma\right>\) where \(\gamma\) is a \textit{\textbf{contraction}}, a function that puts any compact set of \(M\) inside any neighbourhood of any given points after a finite number of iterations. So for example \(\gamma(z)=\frac{1}{2}z\) and then the Hopf manifold consists of the orbits of every point, which are discrete sets within the rays of every point. In fact, every orbit rpeats over and over so that there is one representative in the circle \(S^1\), so that in fact this Hopf manifold is \(S^1 \times S^1\). In general, a Hopf manifold \(H\) is called \textit{\textbf{linear Hopf manifold}}if  \(\gamma\) is linear, and \textit{\textbf{classical Hopf manifold}} if \(\gamma = \lambda \operatorname{Id}\).

\begin{prop}\leavevmode
	A Hopf manifold is diffeomorphic to \(S^1 \times S^{2n-1}\).
\end{prop}

\begin{proof}\leavevmode
If \(H\) is classical, it's simple; if its linear, approximate by classical; in general approximate by linear.
\end{proof}

A \textit{\textbf{Class VII}} surface (also called Kodaira class VII surface) is a complex surface with \(\kappa(M)=\infty\) and first betty number \(b_1(M)=1\). Minimal class VII are called \textit{\textbf{class \(\operatorname{V I I}_0\) surfaces}}.

A \textit{\textbf{primary Hopf surface}} is a Hopf manifold of dimension 2. A \textit{\textbf{secondary Hopf surface is a quotient of a primary Hopf surface \(H\) by a finite group acting freely and holomorphically on \(H\)}}.

\begin{claim}\leavevmode
	Hopf surfaces are class \(\operatorname{V I I}_0\).
\end{claim}

\section{Lecture 2: Hopf manifolds and algebraic cones}

\subsection{Algebraic cones}

\begin{defn}\leavevmode
	Let \(P\) be a projective orbifold (so probably a manifold with mild singularities) and \(L\) an ample line bundle on \(P\). An \textit{\textbf{open algebraic cone}} \(\operatorname{Tot}^0(L)\) is \textbf{just the set of nonzero vectors of the bundle}. 
\end{defn}

In the case of \(P \subset \mathbb{C}P^{n}\) and \(L=\mathcal{O}(1)|_{P}\), the open algebraic cone \(\operatorname{Tot}^0(L)\) can be identified with thw set \(\pi^{-1}(P)\) \textbf{of all \(v \in \mathbb{C}^{n+1}\setminus\{0\}\) projected to \(P\) under the standard map \(\pi:\mathbb{C}^{n+1}\setminus\{0\} \to \mathbb{C}P^{n}\).} The \textit{\textbf{closed algebraic cone}} is its closure in \(\mathbb{C}^{n+1}\). It is an affine subvariety given by the same collection of homogeneous equations as \(P\). Its \textit{\textbf{origin}} is zero.

\begin{thing5}{ChatGPT}\leavevmode
In the case where \(P \subset \mathbb{C}P^n\) and \(L = \mathcal{O}(1)|_P\), the open algebraic cone \(\operatorname{Tot}^0(L)\) can be identified with the set \(\pi^{-1}(P)\), where \(\pi: \mathbb{C}^{n+1} \setminus \{0\} \to \mathbb{C}P^n\) is the standard projection. Explicitly, \(\pi^{-1}(P)\) consists of all \(v \in \mathbb{C}^{n+1} \setminus \{0\}\) that project to points in \(P\).

The \textbf{closed algebraic cone} is the Zariski closure of \(\pi^{-1}(P)\) in \(\mathbb{C}^{n+1}\). It is an affine subvariety defined by the same collection of homogeneous equations as \(P\). Its \textbf{origin} is the zero vector in \(\mathbb{C}^{n+1}\).
\end{thing5}

\begin{thing6}{Hard definition}\leavevmode
An automorphism \(A: P \to P\) is \textit{\textbf{\(L\)-Linearizable}} \(L\) admits an $A$-equivariant structure, in other words, if $A$ can be lifted to an automorphism of the cone \(\operatorname{Tot}^0(L)\) which is linear on fibers.
\end{thing6}

\begin{thing6}{Explanation by ChatGPT}\leavevmode
The definition essentially asks whether \(A\) can be extended to the total space of \(L\) in a way that is consistent with the geometric and algebraic structures of \(L\). This "lifting" ensures that the action of \(A\) on \(P\) interacts harmoniously with the fibers of \(L\).
\end{thing6}

We need that to define \textit{\textbf{Vaisman manifolds}}: they are the quotient \(\operatorname{Tot}^0(L)/\left<A\right>\) where \(A: \operatorname{Tot}^0(L)\to \operatorname{Tot}^0(L)\) which is linear on fibers and satisfies \(|A(v)|=\lambda|v|\) for some number \(\lambda>1\).

Right so notice that Vaisman manifolds and Hopa manifolds are similar. Here's a diagram from the board (from Lecture 3):
\[\begin{tikzcd}
	\operatorname{Tot}^0(L)\arrow[r,hook]\arrow[d,"/\mathbb{Z}"]&\mathbb{C}^N\setminus\{0\}\arrow[d,"/\mathbb{Z}"]\\\text{Vaismann} \arrow[r,hook]&\text{Hopf} 
\end{tikzcd}\]

\begin{quotation}
	Every Vaismann can be embedded to a Hopf-Vaismann (a Hopf that is Veismann). Not any Vaismann is Hopf nor the other way around.
\end{quotation}

\begin{quotation}
	Elliptic non algebraic surfaces are Vaismann
\end{quotation}

\[\begin{tikzcd}
	&  &  \text{Non-algebraic surfaces} \arrow[dll]\arrow[dl]\arrow[d]\arrow[dr]\\
	\text{K3} & \text{ Class VII} &\text{Hopf non-ellptic}&\text{Elliptic (largest class)}  \\
	\text{Not Vaisman} &\text{Not Vaisman} &\text{Sometimes Vaismann} &\text{Vaismann} 
\end{tikzcd}\]




\section{Lecture 3:  Locally conformal Kähler manifolds}

\subsection{Algebraic cones and Vaisman manifolds (reminder)}

\[\begin{tikzcd}
	M \arrow[r,hook]&  \mathbb{C}P^{1}\\
	C_0(M)\arrow[u,"\mathbb{C}^*\text{-fibered} "]&  \mathbb{C}^{n+1}\setminus\{0\}\arrow[u,"\mathbb{C}^0"]\arrow[d,hook]\\
	C(M)\setminus \arrow[ u]\arrow[r,hook]&\mathbb{C}
\end{tikzcd}\]


\subsection{LCK manifolds in terms of differential forms}

So what is Kähler?

\((M,I)\) complex manifolds, $ g$ an \(I\)-invariant Riemannian metric, "Hermitian metric", \(\omega(x,y):=g(Ix,y)\) Hermitian form; \(d \omega=0\) Kähler.

\begin{defn}\leavevmode
	\(\omega\) Hermitian form, \(\omega \in \Lambda^{1,1}_{\mathbb{R}}(M)\), \(\omega(x,Ix)>0\), \(\omega\) is \textit{\textbf{Locally conformally Kähler}} if \(d \omega= \omega \wedge \theta\), \(\theta\) closed 1-form. \(\theta\) is called the \textit{\textbf{Lee form}}.
\end{defn}

\begin{remark}\leavevmode
	The condition if \textbf{conformally invariant}: it is preserved if we replace \(\omega\) by a conformally equivalent form \(f \omega\) for some positive smooth function \(f>0\).Indeed,
	\[d(f \omega)=df \wedge \omega+f d \omega=df \wedge \omega +f \theta \wedge \omega=(df+f\theta)\wedge\omega.\]
\end{remark}

This makes us notice that a classical Hopf manifold \(\frac{\mathbb{C}^n\setminus\{0\}}{\left< \lambda \operatorname{Id}\right>}\) is LCK.

\subsection{Chern connection again}

There is a connection on a holomorphic bundle compatible with the metric that is called \textit{\textbf{Chern connection}}.

The point is that the curvature can be written locally as  \(d d^c\) of some function. And it can be global if you have a non-degenerate holomorphic section taking \(\partial \bar\partial \operatorname{log}|b|\). But it is \(d d^c\) of a function that's the point.

Now there is 

\begin{thing6}{Theorem 5.30}[\cite{verbi}]\leavevmode
The function that maps \(l=\psi:v \mapsto |v|^2\) along with some other stuff like the definition of the function \(q\) then there following expression is true:
\[d d^c l=-q(\theta_B)+\omega_\pi.\]
\end{thing6}

Which leads to

\begin{coro}\leavevmode
	Let \(L\) be a line bundle with negative curvature on a projective manifold. Then the form \(\frac{ d d^c\psi}{\psi}\) is \textbf{homothety invariant} and locally conformally Kähler on \(\operatorname{Tot}^0(L)\).
\end{coro}

\begin{remark}\leavevmode
	We have just shown that \textbf{Vaisman manifolds are LCK}.
\end{remark}

\subsection{Homotheties, monodromy and objective}

We want to give a definition of LCK in terms of a Kähler form on the universal covering. Also might involve local systems. \textbf{Under the alternative definition, LCK manifold is a quotient of a Kähler manifold by a free action of cocompact, discrete group acting by homotheties.}

\begin{claim}\leavevmode
	Any conformal map \(\varphi:(M,\omega) \to (M_1,\omega_1)\) of Kähler manifolds is a homothety.
\end{claim}

\subsection{Reminder on connections and curvature}

The point is that local systems are flat line bundles.

\begin{defn}\leavevmode
	A \textit{\textbf{local system}} on a manifold is a locally constant sheaves of vector spaces.
\end{defn}

\begin{thm}[Rieman-Surfaces lecture 20]\leavevmode
Fix a point \(x \in M\). The category of local systems is naturally equivalent to the category of representations of \(\pi_1(M,x)\).
\end{thm}

\begin{proof}\leavevmode
\begin{enumerate}[label=\textbf{Step \arabic*}]
\item From a locally constant sheaf \(\mathbb{V}\) we construct a vector bundle \(B:=\mathbb{V} \otimes_{\mathbb{R}_M}\mathbb{C}^\infty M\), where \(\mathbb{R}_M\) is the constant sheaf on \(M\). Define a connection \(\nabla\left(\sum_{i=1}^n f_iv_i\right) =\sum df_i \otimes v_i\); where \(v_1,\ldots,v_n\) is a basis in \(\mathbb{V}(U)\). {\color{5}We have constructed a functor from locally constant sheaves to flat vector bundles.}

\item The converse functor takes a flat bundle \((B,\nabla)\) on \(M\) goes to the sheaf of parallel sections \(\nabla b=0\); this sheaf is  locally constant because every vector can be locally extended to a parallel section uniquely (using Frobenius theorem; this is non-trivial).
\end{enumerate}
\end{proof}

\subsection{\(\chi\)-automorphic forms}

The following resembles the way we have define LCK form on a manifold; multipliying by a number something that comes from the universal cover (…?)

\begin{defn}\leavevmode
	Let \(\tilde{M} \xrightarrow{\pi}M\) be the universal covering of \(M\), and \(\xi:\pi_1(M) \to \mathbb{R}^{>0}\) a \textit{\textbf{character}}, which is just a group homomorphism. Consider the natural action of \(\pi_1(M)\) on \(\tilde{M}\). An \textit{\textbf{\(\xi\)-automorphic form}} on \(\tilde{M}\) is a differential form \(\eta \in \Lambda^{k}(M)\) which satisfies \(\gamma^* \eta=\xi(\gamma)\eta\) for any \(\gamma \in \pi_{1}(M)\).

	This makes sense because \(\pi_1(M)\) acts freely on \(\tilde{M}\) (and the quotient is \(M\)), so we can pullback \(\eta\) and it gives an other form on \(\Lambda^{k}(M)\).
\end{defn}

\begin{thing6}{Proposition 1}[What Lada had said!, this is Claim 3.28 \cite{verbi}]\leavevmode
Let \(L\) be a rank 1 local system on \(M\) {\color{3}associated to the representation \(\chi\) (so how is it assocated to \(\chi\)?)}. Then the space of \(\chi\)-automorphic \(k\)-forms on \(\tilde{M}\) is in natural correspondence with the space of sections of \(\Lambda^{k}(M)\otimes L\). Under this equivalence, the de Rham differential on \(\chi\)-automorphic forms corrasponds to the operator \(d_\nabla: \Lambda^{k}(M)\otimes L \to \Lambda^{k+1}(M)\otimes L\).
\end{thing6}

\begin{proof}\leavevmode
\begin{enumerate}[label=\textbf{Step \arabic*}]
\item Pullback the line bundle to the universal cover: \(\tilde{L}:= \pi^*L\), \(\pi:\tilde{M} \to M\). I think \(\tilde{L}\) is trivial: ``The bundle \(\tilde{L}\) is flat and has trivial monodromy, hence it is naturally trivialized by parallel sections".
\end{enumerate}
\end{proof}

\begin{remark}\leavevmode
	\(d_\nabla=d+\theta\)
\end{remark}

\section{Lecture 5: local systems and LCK manifolds}

\subsection{\(\chi\)-automorphic forms again}

\begin{upshot}\leavevmode
	The point is that \(L\)-valued differential forms on \(M\) are in correspondence with \(\chi_L\)-automorphic differential forms \textit{on \(\tilde{M}\)}.
\end{upshot}

\begin{thing6}{Proposition 1}\leavevmode
\((L,\nabla)\) a real flat orientes line bundle. Identify with a local system: associated to \(\chi\) fix a trivializatino of \(L\). Then sections of \(L \otimes \Lambda^{1}(M)\) are in bijection with \(\chi\)-automorphic forms on \(\tilde{M}\) via
\begin{align*}
	\sigma: \Lambda^{\bullet}(M)\otimes L &\longrightarrow \Lambda^{\bullet}(\tilde{M}) 
\end{align*}
\[\sigma(d_\nabla \eta)=d\sigma(\eta).\]
\end{thing6}

\begin{proof}[Very incomplete proof]\leavevmode
\begin{enumerate}[label=\textbf{Step \arabic*}]
\item \(u_1\) a nowhere-vanishing section of \(L\), and \(\theta\) a 1-form such that \(\nabla u_1=u_1 \otimes \theta\).

	\begin{thing7}{Extra}\leavevmode
	How to produce the antiderivative of an exact one form: we integrate from \(x\) to \(y\).
	\end{thing7}
\end{enumerate}
\end{proof}

\subsection{Lichnerowicz cohomology}

Look for  \textbf{Definition 2.51} for definition of \(d_\nabla\), the \textit{\textbf{\(B\)-valued de Rham differential}} of the complex \(\Lambda^{i}(M) \otimes B \longrightarrow \Lambda^{i+1}(M) \otimes B\) given by \(d_\nabla(\eta \otimes b):=d \eta \otimes b +(-1)^{\tilde{\eta}-1}\eta \wedge \nabla b\) for the (real I think) flat bundle \((B,\nabla)\).

\begin{defn}\leavevmode
	Let \(\theta\) be a closed 1-form on a manifold, and \(d_\theta(\alpha):= d \alpha + \theta \wedge \alpha\) be the corresponding differential on \(\Lambda^{*}(M)\). Its cohomology are called \textit{\textbf{Morse-Novikov cohomology}}, or  \textit{\textbf{Lichnerowicz cohomology}}, denoted \(H^{^*}_\theta(M)\).
\end{defn}

\begin{thm}\leavevmode
	Lichnerowitz cohomology of a manifold is equal to the cohomology with coefficientes in a local system defined by \((L,\nabla)\).
\end{thm}

\begin{proof}\leavevmode
Short.
\end{proof}

\subsection{Definition of LCK manifolds in terms of an  \(L\)-calued Kähler form}

\begin{defn}\leavevmode
	Let \((L,\nabla)\) be an oriented real line bundle with flat connection on a complex manifold \(M\), and \(\omega \in L \otimes \Lambda^{ 1,1}(M)\) a \((1,1)\)-form with values in \(L\). We say that \(\omega\) is an \textit{\textbf{\(L\)-valued Kähler form}} if \(\omega(x, Ix) \in L\) is (strictly) positive for any non-zero tangent vector, and \(d_\nabla \omega =0\).
\end{defn}

\begin{thing6}{Superremark}\leavevmode
If we use a trivialization to identify \(L\) and \(C^\infty M\), \(\omega\) becomes a \((1,1)\)-form and \(d_\nabla\) becomes \(d_\theta\), giving \(d_\nabla(\alpha)= d\alpha + \theta \wedge \alpha\). Therefore, \textbf{\(L\)-valued Kähler form on a manifold is the same as an LCK-form}.
\end{thing6}

\subsection{Definition of LCK manifolds in terms of deck transform}

\begin{thing6}{Another definition}\leavevmode
An \textit{\textbf{LCK manifold}} is a complex manifold \(M\), \(\dim_\mathbb{C} M \geq 2\) such that its universal cover \(\tilde{M}\) is equipped with a Kähler form \(\tilde{\omega}\), and the deck transform acts on \(\tilde{M}\) by Kähler homotheties.
\end{thing6}

\begin{thm}\leavevmode
	These two definitions are equivalent.
\end{thm}

\section{Class 6: Vaisman theorem}

\subsection{LCK (reminder)}

\begin{defn}\leavevmode
A complex Hermitian manifold of dimension >1 \((M,I,g,\omega)\) is called \textit{\textbf{locally conformally Kähler}} if there is a closed form \(\theta\) such that \(d \omega =\theta \wedge \omega\). \(\theta\) is the \textit{\textbf{Lee form}} and its cohomology class is the \textit{\textbf{Lee class}}.
\end{defn}

\subsubsection*{The Deal with \(\omega\)}

\paragraph{Fundamental Form:}
\begin{itemize}
    \item The 2-form \(\omega\) is the \textbf{fundamental 2-form} associated with the Hermitian metric \(g\) and the complex structure \(J\), defined by:
    \[
    \omega(X, Y) = g(JX, Y).
    \]
    \item This \(\omega\) is not automatically a \textbf{symplectic form}, because it may fail to be \textbf{closed} (\(d\omega \neq 0\)).
\end{itemize}

\paragraph{Kähler Condition:}
\begin{itemize}
    \item When \(d\omega = 0\), \(\omega\) becomes a symplectic form, and the manifold \(M\) is Kähler. This means \(M\) is simultaneously a symplectic, complex, and Riemannian manifold with a harmonious interaction between these structures.
\end{itemize}

\paragraph{Locally Conformally Kähler:}
\begin{itemize}
    \item In LCK geometry, \(\omega\) doesn’t satisfy \(d\omega = 0\) globally. Instead, it satisfies:
    \[
    d\omega = \theta \wedge \omega,
    \]
    where \(\theta\) is the \textbf{Lee form} (a closed 1-form). This deviation from \(d\omega = 0\) characterizes LCK manifolds.
\end{itemize}

\paragraph{Why Locally Conformal?}
\begin{itemize}
    \item Locally, there exists a function \(f\) such that rescaling the metric by \(e^{-f}\) makes \(\omega\) closed:
    \[
    e^{-f}\omega \text{ is Kähler.}
    \]
    \item This means LCK manifolds are "almost" Kähler but need a conformal adjustment locally.
\end{itemize}

\subsubsection{Some more notes on complex geometry}

That the Kähler form is the differential of a plurisubharmonic function \(\psi\). that is \(\omega=d d^c \psi=\sqrt{-1} \partial \partial \psi\).

And it is a \((1,1)\)-form.

Any positive \((1,1)\) form looks like this:  \(\sum \alpha I x_i \wedge x_i\) for some positive functions \(\alpha_i \geq 0\).

How to prove adjunction formula, that the canonical bundle of a submanifold is the normal bundle of the submanifold tensor product the canonical bundle of the ambient manifold restricted to the submanifold, \(K_M=N_M \otimes K_X |_{M}\): contract a form of the ambient space with a normal section!

\subsection{Vaisman theorem}

Suppose you have a LCK manifold. Suppose you want to replace \(\theta\) by \(\theta'=\theta + df\).
\begin{align*}
d(e^f \omega )&= e^f( d \omega +df \wedge \omega)\\
&=e^f((\theta + df) \wedge \omega),\qquad \text{\(\omega\) is LCK} \\
&=e^f(\theta'\wedge\omega,\qquad \text{\(\theta\) and \(\theta'\) cohomologous} \\
\implies  d \omega'&=(\theta+ df) \wedge \omega'\\
\implies d \omega'&=\theta' \wedge \omega'.
\end{align*}
And the converse is also true:

\begin{quotation}
	conformally equivalent LCK metric give rise to homologous Lee forms, and any closed 1 form cohomologous to the Lee form is a Lee form conformally equivalent LCK metric.
\end{quotation}

\begin{thm}[Vaisman]\leavevmode
	\(M\) LCK, if \([\theta]=0\) (=\(\theta\) is not extact) then \(M\) is not of Kähler type. 
\end{thm}

\begin{remark}\leavevmode
\((M,I)\) Kähler compact: for \(\alpha \in \Lambda^{1}(M,\mathbb{C})\) there exists unique representatives \(\alpha=\alpha^{1,0}+\alpha^{0,1}\) that are closed, i.e.  \(d \alpha^{1,0}=d \alpha^{0,1}=0\).
\end{remark}

\begin{proof}[Proof of Vaisman theorem]\leavevmode
Take a form \(\theta\) and multiply it by its complex conjugate \(I \theta\). I think here we took local coordinates: \(\theta=x_1\) and \(I\theta=y_1\). So \(\omega = \sum x_i \wedge y_i\). And then here's what I don't understand:
\(x_1 \wedge y_1 \left(\sum_{i=1}^n x_i \wedge y_i\right)^{n-1}=(n-1)! x_1 \wedge y_1 \wedge \prod_{i=2}^4 x_i \wedge y_i\). And that's positive!

And that gives the contradiction that
\[0= \int d d^c (\omega^{n-1})>0\]
because any exact form has intergral zero by Stokes.
\end{proof}

\begin{defn}\leavevmode
\textit{\textbf{Vaismann manifold}} is \((M,I,\omega)\) complex hermitian, \(d\omega =\theta \wedge \omega\), \(\nabla\theta=0\), \(\nabla\) Levi-Civita connection of \(g=(\omega(x,Iy)\).

{\color{7}(I think here we may be considering the musical dual of \(\theta\) to take the covariant derivative.)}
\end{defn}

\begin{thing6}{Equivalent definition}\leavevmode
\(G \) complex Lie group acting on an LCK manifold conformally and holomorphically, then  \((M,I)\) is Vaismann.
\end{thing6}

\begin{remark}\leavevmode
The theorem that both definitions are equivalent is hard to prove.
\end{remark}

\subsection{Vaisman examples}
\begin{thm}\leavevmode
Diagonal Hopf is Vaisman.
\end{thm}

Diagonal Hopf is then the \(A\) in \(\mathbb{C}^n/\left<A\right>\) is diagonalizable.

I think

\begin{thm}\leavevmode
Z Hopf. \(Z\) is Vaisman iff is \(Z\) is diagonal.
\end{thm}

\subsection{The fundamental foliation}

\begin{defn}\leavevmode
\(M\) Vaisman manifold, \(\theta ^\sharp\) its Lee fild,, and \(\Sigma\) a 2-dimensional real foliation generated by \(\theta ^\sharp, I\theta ^\sharp\). It is called \textit{\textbf{the fundamental foliation}} of \(M\).
\end{defn}

\begin{question}\leavevmode
So at 
\end{question}

\begin{thm}\leavevmode
\(M\) Vaisman, \(\Sigma\) its canonical foliation.

\begin{enumerate}
\item \(\Sigma\) is independent from the metric (it's canonical).
\item There exists a 2-form \(\omega \in \Lambda^{1,1}(M)\) which is semipositive on every transversal to \(\Sigma\), \(\omega_0|_{\Sigma}=0\), exact.

{\color{6}It's a contact structure, right? On slides: 2. There exists a positive, exact \((1,1)\)-form \(\omega_0\) with \(\sum= \ker \omega_0\)}
	\begin{remark}\leavevmode
	This 2-form is easy to see in these examples: the pullback of Fubini-Study metric: its pullback is exact! Remember that the pullback of any bundle to the \(\operatorname{Tot}^0(M)\) is trivial.
	\end{remark}
\item  \(Z \subset M\) tangent to \(\Sigma\),
\item \(Z \subset M\) is Vaisman.
\end{enumerate}
\end{thm}

\section{Lecture 7: elliptic operators of order 2}

\begin{enumerate}
\item "The ring of symbols"
	\begin{thm}\leavevmode
		\[\bigoplus_{k}\frac{\operatorname{Dif}^k(M)}{\operatorname{Dif}^{k-1}(M)}=\bigoplus_{k}\operatorname{Sym}^{k-m}(TM)\]
	\end{thm}
	So on the lefthandside we have the graded algebra induced by the filtration \(\operatorname{Dif}^0 \subset \operatorname{Dif}^1 \subset \ldots\) of differential operators of different orders on \(M\).

	\begin{proof}\leavevmode
	We give a pairing
	\[\frac{\operatorname{Dif}^k}{\operatorname{Dif}^{k-1}}\otimes \frac{\mathfrak{m}^k}{\mathfrak{m}^{k-1}}\]
and we recall from Hartshorne that \(\frac{\mathfrak{m}^k}{\mathfrak{m}^{k+1}}=\operatorname{Sym}^k(T^*M)\).
	\end{proof}

	\item Definition: \textit{\textbf{symbol}} of the differential operator \(D \in \operatorname{Dif}^k(M)\) is its image in \(\operatorname{Sym}^k(TM)\)
	\item Remark. A differential operator gives us a polynomial function on the cotangent bundle because \(\operatorname{Sym}^k(T^*M)=\) order \(k\) homogeneous polynomial functions on \(T ^* M\).
	\item Definition: \(D \in \operatorname{Dif}^k\) is \textit{\textbf{elliptic}} if \(\sigma(D)\) is positive or negative everywhere on \(T^* M\setminus\{0\}\).
	\item Remark. The symbol of an elliptic operator of second order is positive definite or negative definite. We assume is positive definite. 
	\item 
		\begin{thing6}{Strong maximum principle (version with boundary)}[Hopf]\leavevmode
		\(M\) manifold with boundary. \(D\) elliptic of second order. $f \in C^\infty(M)$  \(D(f) \geq  0\). Then all local maxima of $f$ are on \(\partial M\) or $f$ is constant.
		\end{thing6}
		\begin{proof}[Proof assuming \(D(f)>0\)]\leavevmode
		Because the Hessian is negative semidefinite
		\end{proof}
	\item Weak maximum principle. I think here /cha
		\end{enumerate}

\section{Lecture 8: adjoint operators in Hodge theory}

\subsection{Adjoint connection (reminder)}

First recall that a connection on a vector bundle \(B\) induces a connection on the dual bundle: if \(\nabla: B \to B \otimes \Lambda^{1}(M)\), exists a unique connection \(\nabla^* : B ^*\to  B ^*  \otimes \Lambda^{1}(M)\) satisfying \(d \left< b , \beta\right>=\left<\nabla b, \beta\right>+ \left< b, \nabla ^* \beta\right>\).

The connection \(\nabla ^* \) is called \textit{\textbf{adjoint connection}} to \( \nabla\). The connection \(\nabla = \nabla ^*\) happens precisely when \(\nabla\) preserves the metric tension, consider a section of \( B ^* \otimes B ^* \) and in this case \(\nabla\) is called an \textit{\textbf{orthogonal connection}}.

\subsection{Adjoint connection and \(L^2\)-product}

\begin{upshot}\leavevmode
A scalar product on the space of sections of a vector bundle \(B\). Because we want to define adjoint differential operators on the infinite-dimensional space of sections of the bundle. You multiply sections pointwise, you obtain a function, you integrate that function, you get a number.
\end{upshot}

\(M\) Riemannian manifold, \(b, b'\) sections of \(B\).  \((,)\) scalar product on  \(B\), \((b,b,)_{L^2}=\int(b,b')\operatorname{Vol}\).

\begin{lemma}[Integration by parts]\leavevmode

\end{lemma}
\begin{proof}\leavevmode
The key observation is that
\[\int\mathsf{Lie}_X(\left<b,b'\right>)\operatorname{Vol}=0\]
because \(\mathsf{Lie}_X(\eta)=di_X\eta + \cancelto{0}{i_Xd\eta}\) so \(\mathsf{Lie}_X(\eta)\) is exact.
\end{proof}

\subsection{Adjoint operators}

\begin{defn}\leavevmode
\(A: F \to G\) linear map on vector spaces with scalar products. \(A: G \to G\) is \textit{\textbf{dual}} if \((Ax,y)=(x,A^*y)\). \(A^*y\) is a vector such that \(\left<A^*y,x\right>=\left<Ax,y\right>\).
\end{defn}

Existence is obvious and uniqueness (…)

\begin{claim}\leavevmode
A diferential operator on vector bundles with scalar products \(D : B_1 \to B_2\), then its adjoint \(D^*:B_2 \to B_1\) is also a differential operator of the same order.
\end{claim}

\begin{remark}\leavevmode
Most of you know that \(d^*=\pm  * d *\) which is a composition of linear operators, it's the Hodge star.
\end{remark}

\begin{proof}[Proof of claim]\leavevmode
\(C^\infty M\)-linear. Then we can take the dual point by point (dual exists because its finite dimensional), and it works because it's linear and it's a vector bundle. Then we claim that all first order differential operators are combinations of a fixed connection \(\nabla_X\) (connections are differential operators). Then somehow we have shown that actually \(\nabla_X \) coincides by \(\nabla_X^*\) (integrating by parts).
\end{proof}

\subsection{Laplacian on differential forms}

Start with a Riemannian manifold, say, compact. You can do it with non-compact: if you take care about having things with compact support, and it will work, but we don't want to do it because it takes some extra effort and we won't need it.

Then the sections of \(\Lambda^{*}(M)\) we define scalar product ver naturally:
 \[(\eta,\eta')_{L^2}=\int(\eta,\eta')\operatorname{Vol}_g\]
 now
 \[d^* = \text{dual to $d$} \]
 Also, but this is not related to our course since we did not define Hodge star operator, \(d^*=\pm  * d *\).
 
\begin{defn}\leavevmode
Laplacian is \(\Delta=dd^*+ d^*d\)
\end{defn}

\begin{remark}\leavevmode
It's self dual (self-adjoint) because \(*\) is self dual . And it's positive-definite 
\end{remark}

\section{Lecture 9: Atiyah-Singer index theorem}

Very famous theorem. It's topology. Application of analysis for topology.

\subsection{Fredholm operators}

\begin{defn}\leavevmode
A continuous operator \(F: H_1 \to H_2\) of Hilbert spaces is called \textit{\textbf{Fredholm}} if \textit{its image is closed} ({\color{6}that's new I think}) and its kernel and cokernel are finite dimensional. 
\end{defn}

Next is a condition of invertibility of Fredholm maps. Uses Banach-Schouder theorem. We take the kernel of the map and quotient its domain (we get injectivity); and the restrict the codomain to the image. But we do have to use that BS theorem.

\begin{claim}\leavevmode
\(F: H_1 \to H_2\) is Fredholm iff there is a map \(G:H_2 \to H_1\) such tat \(\operatorname{Id}-FG\) and \(\operatorname{Id}-GF\) have finite rank.
\end{claim}

\begin{defn}\leavevmode
\textit{\textbf{Index}} of \(F\) is \(\dim \ker F-\dim \operatorname{coker}F\).
\end{defn}

This one is also in \cite{brezis} (though not proved there):

\begin{prop}\leavevmode
The class of Fredholm operators is an open subset of \(\mathcal{L}(E,F)\) (with the norm topology, so the norm of an operator is the supremum of its values on the unit sphere).
\end{prop}

So that makes the domain of the index function a reasonable space. Then: \cite{brezis}: the index map \(A \mapsto  \operatorname{ind}A\)is continuous. But not only that:

\begin{thm}\leavevmode
The index function is locally constant.
\end{thm}

So, apparently obviously, the open set of domain of Fredholm operators is not connected, so for example, I think the shift function
So, apparently obviously, the open set of domain of Fredholm operators is not connected, so for example, I think the shift function.

\subsection{Sobolev norm}

\begin{defn}\leavevmode
\(B\) vector bundle with metric over a Riemannian manifold. Define \(L^2_p\) metric for asection \(b \in B\) 
\[|b|^2_p=|b|_{L^2_p}=\sum_{i=0}^p |\nabla^i b|\]
Of course in the non-compact case we must take things with compact support.
\end{defn}

\begin{remark}\leavevmode
This norm is not complete, must take closure.
\end{remark}

\begin{defn}\leavevmode
\(L^2_p\)\textit{\textbf{-topology}} on \(C^\infty M\) is topology defined by \(L^2_p\)-norm.
\end{defn}

\begin{remark}[should be simple]\leavevmode
If \(D\) is a differential operator (on a compact mfld for simplicity) of order \(k\). Then \(D : C^\infty  M,L^2_p \to C^\infty M,L^2_{p-k}\) is (continuous?)

If you have a bound on \(k\) derivatives and you take  more derivatives…
\end{remark}

\subsection{Elliptic operators}

\begin{defn}\leavevmode
\(D: B \to B\) order \(k\) differential operator, then \( \operatorname{symb}(D) \in \operatorname{Sym}^k(T)\otimes \operatorname{End}(B)\) \(k\) (this is agrued using a matrix of symbols) =degree \(k\) polynomial functions on \(T^*M\) with values in \(\operatorname{End}(B)\).

\(D\) is \textit{\textbf{elliptic}} if (after interpreting its symbol as a polynomial function) \(\operatorname{symb}(D)(v)\) is invertible fora ll \(0 \neq  v \in T^*M \).
\end{defn}

\begin{thm}[Elliptic operator is Fredholm (I didn't find it in \cite{brezis})]\leavevmode
\(B_1,B_2\) vector bundles, \(D: B_1 \to B_2\) elliptic of order $p$. Then \(D(B_1,L^2_{k+p}\to (B_2,L^2_k)\) is Fredholm.
\end{thm}

Super hard to prove but super important and basic.

\begin{defn}\leavevmode
\(D: B_1 \to BV_2\) elliptic. Its \textit{\textbf{index}} is the index of the map \(D:(B_1,L^2_p \to (B_2,L^2)\)\end{defn}

\begin{coro}\leavevmode
Let \(D_t\) be a continuous family of elliptic operators. The map \(t \mapsto \operatorname{ind}(D_t)\) is constant. so that if \(D_t\) are elliptic they remain elliptic.
\end{coro}

\begin{proof}\leavevmode
Because we have seen that index is locally constant.
\end{proof}

\subsection{Index theorem for elliptic operators on \(C^\infty M\)}

\begin{thm}\leavevmode
All elliptic operators on \(C^\infty M\) have index 0.
\end{thm}

\begin{proof}\leavevmode
\begin{enumerate}[label=\textbf{Step \arabic*}]
\item \(\operatorname{symb}D\) is a homogeneous function. (by properties of the symbol) we conclude that \(\operatorname{deg}D\) is even.
\end{enumerate}
\end{proof}

\section{Lecture 10: Gauduchon metrics}

\subsection{Positive \((1,1)\) and \((n-1,n-1)\)-forms}

\begin{thing5}{Cornerstone result of linear algebra}\leavevmode
If \(g_0\) is a positive definite scalar product on \(V\) and \(g_1 \in \operatorname{Sym}^2(V^*)\) then there is a basis such that \(g_0\) is the identity matrix and \(g_1\) is a diagonal matrix.
\end{thing5}

Everyone should know the proof, but we won't do it here. Better off, take a hermitian vector space \((V,I)\). We will modify the previous result so that there exists a basis  \(x_1,\ldots,x_n,y_1,\ldots,y_n\) so that \(I(x_i)=y_i\) and \(I(y_i)=x_i\). The basis is orthonormal with respect to some hermitian metric \(g_0\). Then \(\omega_0=g(I \cdot ,\cdot )=\sum_i x_i \wedge y_i\). That's the identity guy. There's also \(\omega_1=g_1(I\cdot ,\cdot )=\sum_i\alpha_i x_i \wedge y_i\) where \(\alpha_i\) are the eigenvalues of \(g_1 \circ g_0^{-1}\). So that's the diagonal guy.

Moving on. \(\Lambda^{1,1}(V)\) is the space of invariant 2-forms. We say \(\omega \in \Lambda^{ 1,1}_{\mathbb{R}}(V)\) is positive if \(\omega(x,Ix)\geq 0\forall x\). It is \textit{\textbf{strictly positive}} if \(>\). Now, using the previous result we see that positivity is equivalent to \(\omega= \sum \alpha_i x_i \wedge y_i \geq 0\).

There is a pairing
\[\Lambda^{1,1}(V)\otimes \Lambda^{n-1,n-1}(V)\to \operatorname{Vol}(V)\]
And looks like also
\[\Lambda^{1,1}(V)\times \operatorname{Vol} \xrightarrow{\sim}\Lambda^{n-1,n-1}(V).\]

\begin{claim}[Equivalences for Positivity for \(n-1,n-1\) forms]\leavevmode
Let \(P \in \Lambda^{n-1,n-1}(V)\). TFAE:
\begin{enumerate}[label=(\roman*)]
\item \(\exists ! z \)such that \(i_z \operatorname{Vol}=P\). \(z\) is positive as a \((1,1)\)-form.
\item This one looks like all we did today, an orthonormal basis, \(Ix_i=y_i\).
\item There's a third one.
\end{enumerate}
\end{claim}

\begin{defn}\leavevmode
An \(n-1,n-1\) form is called \textit{\textbf{positive}} if either of the conditions from the claim hold, and \textit{\textbf{strictly positive}} ``if in the interior".
\end{defn}

Next is an exercise given to everyone and never solved.

\begin{claim}\leavevmode
\((n-1)\)th power of positive form is positive, and moreover, the map \(\alpha \mapsto  \alpha^{n-1}\) defines a homeomorphism (bijective continuous invertible) between strictly positive \((1,1)\) and  \((n-1,n-1)\) forms.
\end{claim}

\begin{proof}\leavevmode

\end{proof}

\begin{remark}[What this is good for]\leavevmode
We have just proved that the map \(\omega \mapsto  \omega^{n-1}\) defines a homeomorphism from the cone (because we can multiply by nonzero positive numbers) of positive \((1,1)\)-forms and the cone of strictly positive \((n-1,n-1)\)-forms.
\end{remark}

\subsection{Harnack inequality}

\begin{thm}[Harnack]\leavevmode
\(L\) elliptic operator, \(\Omega \Subset \Omega_1\), \(L: C^\infty \Omega_1 \to C^\infty \Omega_1\) "Any elliptic eq. has infinitely many solutions". Then \(\exists  C>1\) depending on \(L\), \(\Omega\) and \(\Omega_1\), such that each solution \(Lu=0\) for  \(u\) nonnegative, \(\operatorname{sup}_{\Omega} u \leq  C \operatorname{inf}_{\Omega}U\)
\end{thm}


But we only need the corollary:

\begin{coro}\leavevmode
if \(u \geq 0\) is a solution of \(Lu=0\) then  \(u>0\).
\end{coro}


\subsection{Gauduchon metrics}

\begin{defn}\leavevmode
a hermitian form \(\omega\) on an \(n\)-manifold \(M\) is \textit{\textbf{Gauduchon}} if \(d d^c \omega^{n-1}=0\). So that's a top-form. Well the zero top-form.
\end{defn}

\begin{thm}[Gauduchon]\leavevmode
\(\omega\) hermitian, then there exists a unique up to a constant function \(\psi>0\) such that \(\psi\omega\) is Gauduchon.
\end{thm}

\begin{proof}\leavevmode
So the idea is that \(d d^c\) is always laplacian. It's the trace of…
\end{proof}

\section{Lecture 11: Bott-Chern cohomology and defect of a complex surface}

\subsection{Bott-Chern cohomology}

\begin{defn}\leavevmode
\textit{\textbf{Bott-Chern cohomology}} of a complex manifold \(M\) is \(H^{p,q}_{\operatorname{BC}}(M)\) is the cohomology of \(d d^c\) (I think).
\end{defn}

\begin{remark}\leavevmode
There is no multiplicative structure on BC cohomology.
\end{remark}

\begin{thm}\leavevmode
\(M\) compact complex manifold. \(H^{p,q}_{\operatorname{BC}}(M)\) is finite-dimensional.
\end{thm}

\begin{proof}\leavevmode
Later today.
\end{proof}

\subsection{Elliptic complexes}

\begin{defn}\leavevmode
\textit{\textbf{Elliptic complex}} of vector bundles is when the symbols of the differentials give an exact sequence (I think).
\end{defn}

\begin{defn}[Fredholm complex]\leavevmode

\end{defn}

\begin{coro}\leavevmode
Cohomology of any elliptic complex is finite-dimensional.
\end{coro}

\begin{thm}\leavevmode
BC cohomology of a compact complex manifold is finite-dimensional.
\end{thm}

\begin{proof}\leavevmode
	(This proof also proved finite-dimensionality of Dolbeaut cohomology.)
\end{proof}

\subsection{\(d d^c\) lemma}

\begin{thm}[\(d d^c\) lemma]\leavevmode
\(M\) compact Kähler. \(\eta \in \Lambda^{p,q}(M)\) \(d\)-exact, then \(\eta \in \operatorname{img} d d^c\).
\end{thm}

% Written by ChatGPT
\begin{lemma}[\( d d^c \)-Lemma]
    \textit{(This note was written by ChatGPT.)} 
    Let \( X \) be a compact complex manifold. The following conditions are equivalent for a differential form \( \alpha \):
    \begin{enumerate}
        \item \( \alpha \) is \( d \)-closed and \( d^c \)-exact: \( d \alpha = 0 \) and \( \alpha = d^c \beta \) for some \( \beta \).
        \item \( \alpha \) is \( d^c \)-closed and \( d \)-exact: \( d^c \alpha = 0 \) and \( \alpha = d \gamma \) for some \( \gamma \).
        \item \( \alpha \) is \( d d^c \)-exact: there exists \( \eta \) such that \( \alpha = d d^c \eta \).
    \end{enumerate}
\end{lemma}

\begin{remark}\leavevmode
	\(d d^c\) is equivalent to the natural map \(H^*_{\operatorname{BC}}(M) \to H^{*}(M)\) being injective.
\end{remark}

\subsection{An inequality}

\begin{defn}\leavevmode
\((M,\omega)\) an hermitian surface. \(\eta \in \Lambda^{1,1}(M)\) is called \textit{\textbf{primitive}} if \(\eta \perp \omega\) everywhere.
\end{defn}

\begin{thm}\leavevmode
If \(\alpha \in \Lambda^{1,1}(M)\) is primitive, then
 \[\frac{\alpha \wedge \alpha}{\operatorname{Vol}}=-\|\alpha\|^2 \iff \alpha \wedge \alpha=-\|\alpha\|^2\operatorname{Vol}\]
\end{thm}

\begin{proof}\leavevmode
We can express \(\Lambda^{2}(M)=\Lambda^{+}(M) \oplus \Lambda^{-}(M)\) using the \(1\) and \(-1\) eigenspaces of Hodge star operator. (This is also in \texttt{k3.pdf}; search for ``eigenspeces of the Hodge star operator".)
\end{proof}

\subsection{Top important statement}

\begin{thm}\leavevmode
\(M\) compact complex surface. \(p : H^{1,1}_{\operatorname{BC}}(M)\to H^{2}(M)\) standard map. Then \(\dim \ker p \leq 1\). So that kernel is forms that are exact but not Bott-Chern exact.
\end{thm}

\begin{proof}\leavevmode
	Consider the operator \(D(f)= d d^c f \wedge \omega\) mapping functions to 4-forms. Because \(D\) is elliptic, we get that \(\operatorname{rk} \operatorname{coker}D=1\). Using Gauduchon form; an integral will vanish!
\end{proof}

\subsection{The most important invariant of a surface: the defect}

\begin{coro}\leavevmode
Let \(x \in \ker P\), \(x \neq 0\), then \(\int x\wedge \omega>0\) or \(<0\) for any \(\omega\) Gauduchon. Then a linear combination of \(\omega_1\) and \(\omega_2\) gives a zero integral. But by the previous proof (which I didn't type), the form is exact and it cannot give 0.
\end{coro}

\begin{proof}\leavevmode
Suppose there is \(\omega_1\) giving a positive integral and \(\omega_2\) giving a negative integral. 
\end{proof}

\begin{defn}\leavevmode
The \textit{\textbf{defect}} of a surface  is the number \(\dim \ker P\). It's denoted \(\delta(M)\). By the previous theorem it can only be 0 or 1. We will show that the surface is  Kähler iff \(\delta(M)=1\).
\end{defn}

\section{Lecture 12: Cohomology of a complex surface}

\subsection{Lemma 1 (reminder)}

Looks like the key observation is that the orthogonal complement of \(\omega\) is 3-dimensional. What for? To show that there is an explicit form for \(\Lambda^{+}(M)\) and \(\Lambda^{-}(M)\), namely
\[\Lambda^{+}(M)=\left<\omega_I,\omega_J,\omega_K\right>\qquad \Lambda^{-}(M)=\left<\omega_I,\omega_J,\omega_K\right>^\perp\]
\begin{thing6}{Quote}[Misha]\leavevmode
That \(1,1\) forms orthogonal forms to \(\omega\) are \(\Lambda^{-}(M)\).
\end{thing6}
That is,
\begin{thing5}{Lemma 1}\leavevmode
\[\Lambda^{-}(V)=\{\alpha:I\alpha=\alpha,\alpha \perp \omega\}\]
where \((V,I)\) is given by \(V=\mathbb{R}^n\), \(I^2=-\operatorname{Id}\), \(g\) an \(I\)-invariant scalar product and \(\Lambda^{2}(V)=\Lambda^{+}(V)\oplus \Lambda^{-}(V)\), and of course \(\omega(x,y)=g(Ix,y)\).
\end{thing5}

\begin{quotation}
	It is a useful construction because it allows us to compute because it is an expression of this thing in terms of the complex structure. That's all.
\end{quotation}

\subsection{Reminder on Bott-Chern cohomology}
\[H^{p,q}_{\operatorname{BC}}=\frac{\ker d\Big/ \Lambda^{p,q}(M)}{\operatorname{img} d d ^c}\]
\begin{question}\leavevmode
What's up with the quotient on the numerator?
\end{question}

\subsection{New stuff: intersection form on \(H^{1,1}_{\operatorname{BC}}(M)\)}

\begin{prop}\leavevmode
\(M\) surface with \(\delta(M)>0\).

The intersection form \(\alpha \mapsto  \int\alpha \wedge \alpha\) is negative definite on the image of \(H^{1,1}_{\operatorname{BC}}(M,\mathbb{R})\) in \(H^{2}(M,\mathbb{R})\).
\end{prop}

\begin{proof}\leavevmode
because every cohomology can be represented by something that has degree zero and degree zzro is negative definite.
\end{proof}

And remember that this is a step towards showing that positive defect imples Kähler. Or was it nonKähler?

\subsection{Holomorphic 1-forms on a surface}

\textbf{Introduction.} Surfaces are typically fibrations. Like elliptic fibrations; fibers are tori. So some interesting forms on the surface are pullbacks. And some of this will be holomorphic differentials

\begin{lemma}\leavevmode
All holomorphic 1-forms on a compact complex surface are closed. That is, \(\alpha \in \Lambda^{1,0}(M)\) holomorphic 1-form, then \(d\alpha=0\).
\end{lemma}

\begin{proof}\leavevmode
\(d \alpha= \partial  \alpha \in \Lambda^{ 2,0}(M)\). Then
\[0 < \int d \alpha \wedge d \bar{\alpha} = \int \partial  \alpha = \bar\partial \bar{ \alpha}\].
\end{proof}

\begin{thing3}{Claim 1}\leavevmode
\(\mathcal{H}^{1,1}(M) \oplus  \overline{\mathcal{H}^{1,0}(M)}\hookrightarrow H^{1}(M,\mathbb{C})\)
\end{thing3}

\begin{proof}\leavevmode
Also very simple.
\end{proof}

It will turn out that it is actually isomorphic when defect is zero, and codimension 1 when defect is 1.

\begin{thing6}{Claim 2}\leavevmode
\(R:\overline{\mathcal{H}^{1,0}(M)} \to H^{0,1}_{\bar\partial}(M)\) is injective.
\end{thing6}

\begin{thing7}{Claim 3}\leavevmode
If \(\delta(M)=0\), that same map is surjective.
\end{thing7}

\begin{proof}\leavevmode
If not, we could construct a generator of \(\ker P\), which is impossible.
\end{proof}

Now we can put this all together in one exact sequence; i.e. the following proposition is the three claims put in one. Yes: if \(\ker P=0\) we get some implications right?

\begin{thing4}{Proposition 4}\leavevmode
\[\begin{tikzcd}0\arrow[r]&\overline{\mathcal{H}^{1,1}(M)}\arrow[r,"R"]&H^{0,1}_\partial (M)\arrow[r,"\partial "]&H^{1,1}_{\operatorname{BC}}(M,\mathbb{R})\arrow[r,"P"]&H^{2}(M,\mathbb{R})\end{tikzcd}\]
is exact.
\end{thing4}

{\color{7}This is how we have understood Dolbealt cohomology with respect to defect.} Now let's go to deRham.

\subsection{de Rham cohomology for a complex surface with \(\delta (M)=0\)}

\begin{prop}\leavevmode
The map
\[\tau: H^{1}(M,\mathbb{R})\to H^{0,1}_{\bar{\partial}}(M)\]
taking a close form \(\eta\) to \([\eta^{0,1}]\) is injective.
\end{prop}

\begin{proof}\leavevmode
I was remembering Frölicher spectral sequence during this proof.
\end{proof}

\begin{claim}\leavevmode
\(\tau: H^{1}(M,\mathbb{R})\to H^{0,1}_{\bar{\partial}}(M)\) is sujective \textbf{when defect is zero}, i.e.  \(\delta (M)=0\).
\end{claim}


\begin{thing6}{Dani's toughts}\leavevmode
After going back to all that Frölicher sequence document I wrote once upon a time, and listening to the ideas that are around these lectures, I see that what lies below everything is that old idea of finding harmonic representatives of cohomology classes. Probably that doesn't make sense.

So it looks like these constructions will allow us for distinguishing when the surface is Kähler. But of course we are in surface case! In the Frölicher notes I said: If \(M\) is compact Kähler, there is a harmonic representative of every cohomology class, which says that the Frölicher sequence first page vanishes.

And do recall that Frölicher sequence (so, the convergence of it I suppose) is a statement \textit{similar } to Hodge theorem, which is \(H^{k}(X,C)= \bigoplus_{p+q=k}H^{p,q}(X)\) according to Voisin.
\end{thing6}

\begin{thing7}{Proposition 5}\leavevmode
\(M\) surface \(\delta M=1\) then \(\ker P\) is generated by \(d^c[\theta]\) where \([\theta] \in H^{1}(M,\mathbb{R})\), \(\theta\) closed, \(H^{1}(M,\mathbb{C})=\mathcal{H}^{1,0}(M) \oplus  \overline{\mathcal{H}^{1,0}(M)} \oplus  \left< \theta\right>\).
\end{thing7}

\begin{coro}\leavevmode
\(b_1(M)\) is odd \(\iff \delta(M)=1\)
\end{coro}

\begin{coro}\leavevmode

\end{coro}

\subsection{Frölicher spectral sequence}

\begin{defn}\leavevmode
\(M\) complex compact. We say that \textit{\textbf{Frölicher-HdR degenerates in \(E^{p ,p}_1\)}}  degenerates isf any \(\alpha \in H^{p,q}_{\bar{\partial}}(M)\) can be represented by \(\bar\partial\)-closed \(\alpha\) such that \(\partial  \alpha \in \operatorname{img} \bar\partial\).
\end{defn}

\begin{remark}\leavevmode
In Kähler manifolds, you have Hodge theory, so you have degeneration (Dani: I think he means that you have the harmonic representative).
\end{remark}

So it looks like the result is that that happens for complex surfaces anyways?

\begin{coro}\leavevmode
Hodge de Rham F spectral sequence degenerates on complex surface in \(E^{1,0}_0\) and \(E^{0,1}_1\).
\end{coro}

So I'd say:
\begin{quotation}
	It's not that we have Hodge theory, but we have something similar. (But that's just me.)
\end{quotation}

\begin{thing4}{Quote}[Misha during the proof of the corollary]\leavevmode
…so we have that result, that image of \(\partial \) is image of \(\bar\partial\)
\end{thing4}

\section{Lecture 13: Currents}

\subsection{Currents and generalized functions}

Currents should be funcionals on volume forms, right? Because the latter a like functions.

\begin{defn}\leavevmode
\textit{\textbf{Generalized functions (distributions)}} on manifold \(M\) is a functional on functions with compact support, which is continuous in \textit{one}  the \(C^i\) topologies (so probably that means it is continuous in \(C^\infty\) topology), where..
\end{defn}

\begin{defn}\leavevmode
Let \(F\) be a Hermitian bundle with connection \(\nabla\), on a Riemannian manifold \(M\) with Levi-Civita connection, and
\[{\color{6}\|f\|_{C^k}:=\operatorname{sup}_{x \in M}\Big(|f|+|\nabla f|+\ldots+|\nabla^kf|}\]
the corresponding \({\color{6}C^k}\) \textit{\textbf{norm}}. Defined \textbf{on smooth sections with compact support}. The \(C^k\) topology is independent from the choice of connection and metrics.
\end{defn}

\begin{remark}[Misha]\leavevmode
as \(k\) grows there are ( I think) \textit{less} convergent sequence, which means that the set of continuous functional is larger. Yes because if a sequence converges in \(C^k\) then it converges in all \(\ell \leq k\).
\end{remark}

Right so back to currents, a top form \(v \in \Lambda^{n}(M)\) gives a functional via \(\left<v,f\right>=\int vf\). So also you can just do \(\alpha \in \Lambda^{k}(M)\) and do the functional \(\left<\alpha,\eta\right>=\int \alpha \wedge \eta\) for \(\eta \in \Lambda^{ n-k}(M)\). \textbf{This embeds currents into differential forms.} Or the other way around, yes,

\begin{defn}\leavevmode
\textit{\textbf{Current}} is functional on \(k\)-forms with compact support, continuous in \(C^i\) for \textit{some} \(i\). The space of currents is \(D^k(M)\). 
\end{defn}

\begin{remark}\leavevmode
A volume form gives an isomorphism \(D^0 \cong D^n\).
\end{remark}

\begin{upshot}[The way to think about currents]\leavevmode
Currents are the same as differential forms wiht coefficients in generalized functions:
\[D^0 \otimes_{C^\infty M}\Lambda^{k}(M) \cong D^k(M).\]
\end{upshot}

So we even get a product between forms and currents.

\subsection{Currents on complex manifolds}

\begin{defn}\leavevmode
There is a \textit{\textbf{weak topology}} on currents: sequence converges iff it converges on all forms with compact support.
\end{defn}

\begin{claim}\leavevmode
De Rham differential is continuous on currents and the cohomologies are the same.
\end{claim}

The point is that currents gives you resolutions in the same way as de Rham and Dolbeault cohomologies do.

\subsection{Positive forms again}

Do you know this by now? It's a 2-form that the fundamental form associated is a positive (definite? Don't be confused with the french) form. But not only a 2-form: a 1,1 form! Why? From home assignment 5 of k3 course I once convinced myself that 1,1 forms are those that preserve the complex structure in the sense that: \(\eta(Ix,Iy)=\eta(x,y)\)… so probably we need \(1,1\) condition to get that the ``fundamental form" associated to \(\eta\) \textbf{is a riemannian metric}.

\subsection{Riesz representation theorem}

\begin{defn}\leavevmode
A \textit{\textbf{Borel measure}} is generated from the open sets. It is \(\sigma\)-additive (I think this is good behavious against countable unions). It is called \textit{\textbf{Radon measure}} if it is finite on any compact set.
\end{defn}

\begin{thm}[Riesz representation theorem]\leavevmode
Let \(M\) be a metrizable, locally compact topological space, \(C^0_c(M)\) the space of continuous functions with compact support, and \(C^0_c(M)^*\) the space of functionals continuous in uniform topology. Then the Radon measures on \(M\) can be characterized as functionals \(\mu \in C^0_c(M)^*\) which are non negative on all non-negative functions.
\end{thm}

\begin{proof}\leavevmode
 One side is obvious: every measure is a functional. To construct a measure from a functional we do it on compact sets.
\end{proof}

\subsection{Positive currents}

\begin{defn}\leavevmode
\(\xi \in D^{1,1}_\mathbb{R}(M)\) is \textit{\textbf{positive}} if \(\left<\xi,\tau\right> \geq 0\) \(\forall \tau\) positive with compact support.
\end{defn}

\begin{thing7}{Super upshot}[What is a measure on a manifold]\leavevmode
measures are sections of \(D^n(M)=D^0 \otimes \operatorname{Vol}\) where \(\operatorname{Vol}\) is determinant bundle (I'm pretty sure).
\end{thing7}

\section{Lecture 15: plurisubharmonic}

\subsection{Intro}

\[\begin{tikzcd}
&\tilde{M}\arrow[dl,"",swap]\arrow[dr]\\
M_1\arrow[rr]&&M_2
\end{tikzcd}\]
Proper holomorphic cover, then \(M_1\) kahler iff \(M_2\) Kähler if \(\dim M \leq  2\).

Proof uses Kähler current: \(\eta \in D^{1,1}(M,\mathbb{R})\), \(d\eta=0\) with \(\eta- \varepsilon \omega \geq 0\) for some hermitian form \(\omega\) (it's not only positive, but more positive than \(\omega\)).

\subsection{Poincaré lemma for $\bar\partial$, ddc lemma}

\begin{defn}\leavevmode
plurisubharmonic is \(f\) if \(dd^c\) is a positive 1 1 form. strictly if  \(d d ^c\) is Kähler.
\end{defn}

\begin{lemma}[Poincaré-Grothendieck-Dolbeault]\leavevmode
	\(\bar\partial\) closed implies \(\bar\partial\) exact.
\end{lemma}

\begin{coro}[\(d d ^c\) lemma for dim 2 (local ddc lemma?)]\leavevmode
\(\omega \in \Lambda^{ 11}(\Big).\) \(d\omega=0 \implies  \omega=d d ^c f\).
\end{coro}

This ddc lemma says that a closed form is ddc of a function!

\subsection{Subharmonic functions}

\begin{defn}\leavevmode
	\(f \in C^\infty(M,\mathbb{R})\) is called \textit{\textbf{upper semicontinuous}} if \(\operatorname{ l i m s u p}_{z \to z_0} f(z) \leq f(z_0) \). [From René Baire thesis]

A measurable [upper semicontinuous, integrable] function \(f \in C^\infty(\Omega,\mathbb{R})\), \(\Omega \subset M\), is called \textit{\textbf{subharmonic}} if for every open ball with center in \(z_0\) we have that the average of \(f\) in this ball is \(\leq f(z_0)\).

\textit{\textbf{psh}} if it is  sh on all lines of \(\mathbb{C}^n\).
\end{defn}

\begin{thm}[trivial with nontrivial implication]\leavevmode
convex function \(\chi: \mathbb{R}^n \to \mathbb{R}\), convex, monotonously nondecreasing in each variable, and \(u_1,\ldots,u_k : \Omega \to \mathbb{R}\) sh, then the composition \(\chi(u_1,\ldots,u_k\) is sh.
\end{thm}

\begin{proof}\leavevmode
We approximate with affine functions.
\end{proof}

\begin{coro}\leavevmode
If \(u_1,\ldots, u_n\) is psh then so is the sum, the max, and \(\operatorname{log}\left(\sum \operatorname{exp}(u_i)\right) \)
\end{coro}

\subsection{Pullback and pushforwards}

\begin{defn}\leavevmode
\(f: X \to Y\) smooth proper map. Define
\[f:D^k(X) \to D^{k-r}(Y)\]
,where \(r = \dim X - \dim Y\), 
\[\left<f_*\phi,\tau\right>=\left<\eta,f^*\tau\right>\]
where \(\tau\) is any test-form.
\end{defn}

\begin{claim}\leavevmode
if \(f\) is a proper submersion then the pushforward of a differential form is a differential form.
\end{claim}

\subsection{Smoothing kernels}

\begin{defn}\leavevmode
\(\mu_i: \mathbb{R}^n \to \mathbb{R}^{>0}\) with support on \(B_{r_i}\), a ball with radius \(r_i\) and center in 0. Also \(\int_{\mathbb{R}^n} \mu_i \operatorname{Vol}=1\). \(\mu_i\) is called \textit{\textbf{family of smoothing kernels}} if \(r_i \xrightarrow{i \to \infty}0\).

\textit{\textbf{Convolution with respect to \(\mu_i\)}} is \(\eta \mapsto  \eta \star\mu_i\). Consider \(\pi_1,\pi_2,\pi_\Sigma\) projections \(\mathbb{R}^n \times \mathbb{R}^n \to \mathbb{R}^n\) and \(\pi_\Sigma(x,y)=x+y\). We define
\[\eta \star \mu_i:=(\pi_2)_*(\pi_1^*\eta\wedge\pi_\Sigma ^*(\mu_i\operatorname{Vol})\]
so its a map that maps a \(k\)-form to a \(k\)-form.
\end{defn}

\begin{claim}\leavevmode
\begin{enumerate}
\item \(\lim_{ \eta\star \mu_i} =\eta\).
\item \(\eta \star \mu_i\) smooth (when \(\eta\) is invertible cofficient)
\item \(\eta \to \eta \star \mu_i\) commutes with de Rham, \(\partial ,\bar\partial\)
\end{enumerate}
\end{claim}

\begin{proof}\leavevmode
It commues with de Rham because pullback, multiplication by closed form and pushforward also do.
\end{proof}

\begin{thm}\leavevmode
singular sh function is limit of smooth functions.
\end{thm}

\section{Lecture 16}

\begin{defn}\leavevmode
A \textit{\textbf{family of smoothing kernels}} is a family of measures \(\mu_i\) on \(\mathbb{R}^n\) with compact support in an open ball \(B_{r_i}\) with center in \(0\), such that \(\int_{\mathbb{R}^n}\mu_i=1\)
\end{defn}

\begin{claim}\leavevmode
\(\mu \star \mu_i\) is smooth and \(\lim_{i} (\eta \star\mu_i)=\eta\), and commutes with de Rham
\end{claim}

\begin{thm}\leavevmode
\(v \) subharmonic function on a ball \(B_1 \subset \mathbb{R}^n\) centered in 0. Then the average monotonously decreases to \(v(0)\) as  \(r\) decreases to 0, ie. \(r \to \operatorname{ A v}_{B_r}v\) monotonously.
\end{thm}

\begin{proof}\leavevmode
Using Green representation formula from homework 3.
\end{proof}

\begin{remark}\leavevmode
Think of \(\operatorname{log}|z|\). It is subharmonic and its value at zero is \(- \infty\).

Also think of \(\frac{1}{r^{n-1}}\) on \(\mathbb{R}^n\)
\end{remark}

\begin{remark}\leavevmode
In \(\mathbb{R}\), subharmonic is convex.
\end{remark}

\subsection{A nice family of smoothing kernels}

The family is
\[\mu_\varepsilon=\varepsilon^{-1}h_\varepsilon ^*\mu_1\]
where \(\mu_1\) is a nonnegative smooth function on \(\mathbb{R}^n\) which \(\int_{\mathbb{R}^n}\mu_1\operatorname{Vol}=1\), has supprt on open ball.

\begin{thm}\leavevmode
 \(u\) subharmonic on \(\Omega \subset \mathbb{R}^n \) then \(u \star \mu_\varepsilon \) is monotonous subharmonic and converges to \(u\). In particular, all subharmonic functions are obtained as monotnonous limits of smooth subharmonic functions.
\end{thm}

\begin{proof}\leavevmode
\(u \star \mu_\varepsilon\) is clearly subharmonic.
\end{proof}

\subsection{Plurisubharmonic functions: singular case}

I think Oka was isolated.

\begin{defn}[Lelong, Oka, 1942]\leavevmode
\(f\) on \(\Omega \subset \mathbb{C}^n\) semicontinuous, measurable (can take values \(-\infty\) but not \(\infty\)), \(f\) restricted to complex line is subharmonic. Then \(f\) is psh.
\end{defn}

\begin{exercise}\leavevmode
It's almost obvious: \(d d^c f |_{\text{line} }=\Delta\). Just write in coordinates.
\end{exercise}

\begin{claim}\leavevmode
Decreasing limit of psh is psh.
\end{claim}

\begin{claim}\leavevmode
\(\sum\) psh is psh, also max and \(\operatorname{log}(\operatorname{exp}(\text{psh} )\).
\end{claim}

\begin{claim}\leavevmode
same family of smooth kernels, \(f\) psh, then \(f \star \mu_2\) is psh smooth and converges monotnonously.
\end{claim}

\section{Lecture 17: Poincaré-Lelong formula and regularization of currents}

\subsection{Intro}

We want to solve \(\Delta G=\delta_0\). Of course you could put \(dd^cG=\delta_0\).

Or maybe try to solve \(\Delta f =g\).

So the method is this: you compute this integral
\[\int_{\mathbb{R}^n}L_xGg\operatorname{Vol}\]
and realise it it gives you 
\[g\]
And also, it is just
\[\int L_x \delta_0 g \operatorname{Vol}.\]
And remember that \(L_x\) is, like in past lectures, translation about \(x \in \mathbb{R}^n\).

\subsection{Cauchy formula}



\begin{thm}[Cauchy formula]\leavevmode
\(f \in C^1(\Delta)\). Then \(\forall w \in \Delta\) such that
\[f(w)=\frac{1}{2\pi \sqrt{-1}}\int_{\partial  \Delta}\frac{f(z)}{z-w}dz{\color{6}-\int_\Delta \frac{ 1}{\pi(z-w)}\frac{\partial  f}{\partial  \bar{z}}\operatorname{Vol}}\]
where \(\operatorname{Vol}=dz \wedge d\bar{z}\) is the standard volume form.
\end{thm}

So there is an extra term that vanishes when \(f\) is holomprhic.

\subsection{Poincaré-Lelong formula}

\begin{upshot}\leavevmode
This says that \(dd^c \operatorname{log}|f|\) is psh.
\end{upshot}

\begin{thing5}{Poincaré formula}\leavevmode

	\(f\) holomorphic on any complex manifold \(M\). Then \(dd^c \operatorname{log} |f|=\frac{1}{4\pi}[Z]\). (or maybe its \(2\pi\).) Where \(Z\) is the zro set of  \(f\). So  \([Z]\) is the integration current, meaning (by definition of integration current)
	\[\left<[Z],\tau\right>=\int_Z \tau\]
\end{thing5}

First, more simple case we proved in class:

\begin{prop}[Poincaré-Lelong formula on \(\mathbb{C}\)]\leavevmode
Consider the function \(\ell(z):= \operatorname{log} | z|\) on a disk \(\Delta \subset \mathbb{C}\). Then \(\ell\) is psh, and \(dd^c \ell=\delta_0\) (defined as a current \(\left<\delta_0,\tau\right>=\tau_0\).
\end{prop}

General version:

\begin{thm}[Poincaré-Lelong formula]\leavevmode
	(What is Poincaré contribution here?) \(f\) holomorphic, then \(\operatorname{log}|f|\) is psh, moreover \(dd^c |f|=\frac{1}{2\pi}[D_f]\) where \([D_f]\) is the integration current of zero divisor of \(f\). (We assume that \(0\) is a regular value.)
\end{thm}

\begin{thing6}{Idea}\leavevmode
Every closed subscheme on a complex algebraic variety corresponds to some psh function.
\end{thing6}

\begin{coro}\leavevmode
\(f_1,\ldots,f_n\) collection of holomorphic functions on a complex manifold. Then \(\operatorname{log}\left(\sum_{i}e^{u_i}\right) \) is psh.
\end{coro}

\begin{proof}\leavevmode
By PL lema, \(u_i=\operatorname{log}|f_i|^2\) is psh. Then \(\operatorname{log}\left(\sum_{i}|f_i|^2\right) =\operatorname{log}\left(\sum_{i}e^{u_i}\right) \) is also psh from past lecture.
\end{proof}

\subsection{Regularized maximum}
\begin{upshot}\leavevmode
We shall learn how we deal with singularities with currents. The Regularized maximum is something that is smooth.
\end{upshot}

In reality I missed what regularized maximum means. Input is two functions.

\begin{remark}\leavevmode
\(\mu:\mathbb{R}^2 \to \mathbb{R}\) convex, monotonous functions in both arguments. Then \(f_1,f_2\) psh \(\implies  \mu(f_1,f_2)\) is psh.
\end{remark}

\begin{defn}\leavevmode
\textit{\textbf{nef current}} is a limit of smooth, closed, positive currents.
\end{defn}

\begin{remark}\leavevmode
A curve is nef in AG sense iff its current of integration is nef. (or maybe only one directino works)
\end{remark}

\subsection{Lelong sets}

Take a positive current \(\Theta\) on \(M\), and a point \(x \in M\). And let \(\eta_x=dd^c \operatorname{log}\operatorname{dist}x\)

We want to define

\begin{defn}\leavevmode
\textit{\textbf{Lelong number}} \(\nu_x(\theta)\) is a measure \(\Theta \wedge (\eta_x)^{n-p}\).

\(\left<\mu,x\right>:= \nu_x\theta\)
\end{defn}

There is a way to decompose (I think any) measure into two parts.

\begin{defn}\leavevmode
\textit{\textbf{Lelong set}} for \(s>0\) is  
\[Z( \Theta,s):=\{x \in M : \nu_x(\theta)>s\}\]

\end{defn}

\begin{thm}[Y.T. Siu, 1974]\leavevmode
For any \(\Theta\) and any \(s>0\), the Lelong set \(Z_s\) is complex analytic.
\end{thm}

\subsection{The multiplier ideals}

\begin{defn}\leavevmode
\(f\) a function on \(M\), locally sum of smooth and psh, then \(f\) is called \textit{\textbf{almost psh}}.
\end{defn}

\begin{defn}\leavevmode
If you have any line bundle and \(h\) is a metric and \(f\) is almost psh, then \(h e^{-f}\) is called \textit{\textbf{singular metric}} on \(L\).
\end{defn}

\begin{defn}\leavevmode
 \((L,e^{-f}h\) line bundle with singular metric, \(I^L_f\) a sheaf of holomorphic sections of \(L\), sections \(L^2\)-integrable, then
 \[I^L_f \otimes L^{-1} \subset L \otimes L^{-1} \subset \mathcal{O}_M\]
 
\end{defn}

\begin{thm}[Nadel]\leavevmode
It is a coherent sheaf.
\end{thm}

\begin{upshot}\leavevmode
support of multiplier ideal is the same as Lelong set.
\end{upshot}


\subsection{Demailly's regularization theorem}

\begin{defn}\leavevmode
\(\eta\) is closed \((1,1)\)-current, \(\eta\) has \textit{\textbf{algebraic singularities}} if \(\eta=dd^c \operatorname{log} \sum |f_i|^2+\eta_0\), \(f_i\) smooth, \(\eta_0\) smooth
\end{defn}

\begin{upshot}\leavevmode
How to approximate currents with something reasonable: these algebraic (logarithmic) singularities I think they are reasonable.
\end{upshot}

\begin{thm}[Demailly Regularization Theorem]\leavevmode
	\(T\) positive closed \((1,1)\)-current on a Kähler manifold \((M,\omega)\) then \(T\) is a limit of currents \(T_k\) with algebraic singularities. And also there exists a sequence \(\varepsilon_i\) that converges to zero and such that \(T_k+ \varepsilon_k \omega\) is positive. And of course  all  \(T_i\) are in the same cohomology class. 

	Finally, for all \(x \in M\), \(\lim \nu_x T_k = \nu_x T\) and convergence is monotonous.
\end{thm}

\begin{thing6}{Note}\leavevmode
You can google Demailly Regularization to see all this happening.
\end{thing6}

\begin{remark}\leavevmode
The theorem is super difficult.
\end{remark}

And what we need here is

\begin{coro}\leavevmode
Current \(T\) with zero Lelong numbers is nef. Which means it is a limit of smooth positive closed.
\end{coro}










%\bibliography{bib.bib}
\end{document}
