\input{/Users/daniel/github/config/preamble.sty}%available at github.com/danimalabares/config
\input{/Users/daniel/github/config/thms-eng.sty}%available at github.com/danimalabares/config

\setcounter{secnumdepth}{2}
\bibliographystyle{alpha}
\begin{document}

\begin{minipage}{\textwidth}
	\begin{minipage}{1\textwidth}
		Complex Surfaces \hfill Daniel González Casanova Azuela
		
		{\small Prof. Misha Verbitsky\hfill\href{https://github.com/danimalabares/sur}{github.com/danimalabares/sur}}
	\end{minipage}
\end{minipage}\vspace{.2cm}\hrule

\vspace{10pt}
{\huge Home assignment 3: Harmonic functions}

Comments after meeting with Lada:
\begin{itemize}
\item \(d^* =(-1)^{n(k+1)+1}* d *\).
\item \(\Delta f= * d * df\)
\end{itemize}

\textbf{How to prove mean value for harmonic:} you use exercise 8 to realise that the average of the function on spheres of different radius remains the same. Then you compute the average on the ball as an integral of smaller radius of the integrals on the sphere. Integrating in really small spheres will take you to the value on the point.

\end{document}
