\input{/Users/daniel/github/config/preamble.sty}%This is available at github.com/danimalabares/config
\input{/Users/daniel/github/config/thms-eng.sty}%This is available at github.com/danimalabares/config

%\usepackage[style=authortitle-terse,backend=bibtex]{biblatex}
%\addbibresource{/Users/daniel/github/config/bibliography.bib}

\begin{document}
\bibliographystyle{alpha}

\begin{minipage}{\textwidth}
	\begin{minipage}{1\textwidth}
		 Complex Surfaces\hfill Daniel González Casanova Azuela
		
		{\small Prof. Misha Verbitsky\hfill\href{https://github.com/danimalabares/}{github.com/danimalabares/}}
	\end{minipage}
\end{minipage}\vspace{.2cm}\hrule

\vspace{10pt}
{\huge Complex surfaces}

\tableofcontents

\section{Lecture 1: Kodaira dimension and Hopf manifolds}

\subsection{Outline}
\begin{enumerate}
\item Kodaira dimension definition: the function \(P(N)=H^{0}(K^N)\) is polynomial (so probably \(h\) rathen than \(H\) right?). The degree  \(\kappa(M)\) is called \textit{\textbf{Kodaira dimension}} of \(M\). If \(P\) is identically 0, we set \(\kappa(M)=-\infty\).
\item Nilmanifolds and solvmanifolds (quotients of (solvable) Lie groups.
\item Kodaira surface definition (it's a nilmanifold).
\item Minimal models. A complex surface is \textit{\textbf{minimal}} if it does not contain a smooth rational curve with self-intersection \(-1\). Thorem by Hartshorne: for any complex surface \(M\) there exists a minimal surface \(M_1\) and a holomorphic, bimeromorphic map \(M \to M_1\). \(M_1\) is called a \textit{\textbf{minimal model for \(M\).}}
\item Kodaira theorem: a complex surface is projective iaotfh: (i) field o meromorphic functions has trascendental dimension 2, (ii)  \(M\) admits a holomorphic line bundle \(L\) with \(c_1(L)^2>0\), (iii) the Neron-Severy lattice of \(M\), \(\operatorname{NS}(M):=H^{1,1}(M)\cap H^{2}(M,\mathbb{Z})\), contains a class with positive self-intersection. 
\item Class VII and \(\operatorname{V I I}_0\) surfaces definition.
\item Hopf manifolds (Hopf manifolds are \(\operatorname{  V I I}_0\)).
\end{enumerate}

\subsection{Kodaira-Enriques classification for non-algebraic surfaces: constructions and examples}

\begin{itemize}
\item \textit{\textbf{(Primary) Kodaira surface}} can be defined as \(M:=G/\Gamma\) with the complex structure defined by the subalgebra \(\mathfrak{g}^{1,0}:=\left<x+\sqrt{-1}y,z+\sqrt{-1}t\right>\), which is actually abelian.
\end{itemize}

\subsection{Holomorphic contractions and Hopf manifolds}

\textit{\textbf{Hopf manifolds}} are quotients \(\mathbb{C}^\setminus\{0\}/ \left<\gamma\right>\) where \(\gamma\) is a \textit{\textbf{contraction}}, a function that puts any compact set of \(M\) inside any neighbourhood of any given points after a finite number of iterations. So for example \(\gamma(z)=\frac{1}{2}z\) and then the Hopf manifold consists of the orbits of every point, which are discrete sets within the rays of every point. In fact, every orbit rpeats over and over so that there is one representative in the circle \(S^1\), so that in fact this Hopf manifold is \(S^1 \times S^1\). In general, a Hopf manifold \(H\) is called \textit{\textbf{linear Hopf manifold}}if  \(\gamma\) is linear, and \textit{\textbf{classical Hopf manifold}} if \(\gamma = \lambda \operatorname{Id}\).

\begin{prop}\leavevmode
	A Hopf manifold is diffeomorphic to \(S^1 \times S^{2n-1}\).
\end{prop}

\begin{proof}\leavevmode
If \(H\) is classical, it's simple; if its linear, approximate by classical; in general approximate by linear.
\end{proof}

A \textit{\textbf{Class VII}} surface (also called Kodaira class VII surface) is a complex surface with \(\kappa(M)=\infty\) and first betty number \(b_1(M)=1\). Minimal class VII are called \textit{\textbf{class \(\operatorname{V I I}_0\) surfaces}}.

A \textit{\textbf{primary Hopf surface}} is a Hopf manifold of dimension 2. A \textit{\textbf{secondary Hopf surface is a quotient of a primary Hopf surface \(H\) by a finite group acting freely and holomorphically on \(H\)}}.

\begin{claim}\leavevmode
	Hopf surfaces are class \(\operatorname{V I I}_0\).
\end{claim}

\section{Lecture 2: Hopf manifolds and algebraic cones}

\subsection{Algebraic cones}

\begin{defn}\leavevmode
	Let \(P\) be a projective orbifold (so probably a manifold with mild singularities) and \(L\) an ample line bundle on \(P\). An \textit{\textbf{open algebraic cone}} \(\operatorname{Tot}^0(L)\) is \textbf{just the set of nonzero vectors of the bundle}. 
\end{defn}

In the case of \(P \subset \mathbb{C}P^{n}\) and \(L=\mathcal{O}(1)|_{P}\), the open algebraic cone \(\operatorname{Tot}^0(L)\) can be identified with thw set \(\pi^{-1}(P)\) \textbf{of all \(v \in \mathbb{C}^{n+1}\setminus\{0\}\) projected to \(P\) under the standard map \(\pi:\mathbb{C}^{n+1}\setminus\{0\} \to \mathbb{C}P^{n}\).} The \textit{\textbf{closed algebraic cone}} is its closure in \(\mathbb{C}^{n+1}\). It is an affine subvariety given by the same collection of homogeneous equations as \(P\). Its \textit{\textbf{origin}} is zero.

\begin{thing5}{ChatGPT}\leavevmode
In the case where \(P \subset \mathbb{C}P^n\) and \(L = \mathcal{O}(1)|_P\), the open algebraic cone \(\operatorname{Tot}^0(L)\) can be identified with the set \(\pi^{-1}(P)\), where \(\pi: \mathbb{C}^{n+1} \setminus \{0\} \to \mathbb{C}P^n\) is the standard projection. Explicitly, \(\pi^{-1}(P)\) consists of all \(v \in \mathbb{C}^{n+1} \setminus \{0\}\) that project to points in \(P\).

The \textbf{closed algebraic cone} is the Zariski closure of \(\pi^{-1}(P)\) in \(\mathbb{C}^{n+1}\). It is an affine subvariety defined by the same collection of homogeneous equations as \(P\). Its \textbf{origin} is the zero vector in \(\mathbb{C}^{n+1}\).
\end{thing5}

\begin{thing6}{Hard definition}\leavevmode
An automorphism \(A: P \to P\) is \textit{\textbf{\(L\)-Linearizable}} \(L\) admits an $A$-equivariant structure, in other words, if $A$ can be lifted to an automorphism of the cone \(\operatorname{Tot}^0(L)\) which is linear on fibers.
\end{thing6}

\begin{thing6}{Explanation by ChatGPT}\leavevmode
The definition essentially asks whether \(A\) can be extended to the total space of \(L\) in a way that is consistent with the geometric and algebraic structures of \(L\). This "lifting" ensures that the action of \(A\) on \(P\) interacts harmoniously with the fibers of \(L\).
\end{thing6}

We need that to define \textit{\textbf{Vaisman manifolds}}: they are the quotient \(\operatorname{Tot}^0(L)/\left<A\right>\) where \(A: \operatorname{Tot}^0(L)\to \operatorname{Tot}^0(L)\) which is linear on fibers and satisfies \(|A(v)|=\lambda|v|\) for some number \(\lambda>1\).

Right so notice that Vaisman manifolds and Hopa manifolds are similar. Here's a diagram from the board (from Lecture 3):
\[\begin{tikzcd}
	\operatorname{Tot}^0(L)\arrow[r,hook]\arrow[d,"/\mathbb{Z}"]&\mathbb{C}^N\setminus\{0\}\arrow[d,"/\mathbb{Z}"]\\\text{Vaismann} \arrow[r,hook]&\text{Hopf} 
\end{tikzcd}\]

\begin{quotation}
	Every Vaismann can be embedded to a Hopf-Vaismann (a Hopf that is Veismann). Not any Vaismann is Hopf nor the other way around.
\end{quotation}

\begin{quotation}
	Elliptic non algebraic surfaces are Vaismann
\end{quotation}

\[\begin{tikzcd}
	&  &  \text{Non-algebraic surfaces} \arrow[dll]\arrow[dl]\arrow[d]\arrow[dr]\\
	\text{K3} & \text{ Class VII} &\text{Hopf non-ellptic}&\text{Elliptic (largest class)}  \\
	\text{Not Vaisman} &\text{Not Vaisman} &\text{Sometimes Vaismann} &\text{Vaismann} 
\end{tikzcd}\]




\section{Lecture 3:  Locally conformal Kähler manifolds}

\subsection{Algebraic cones and Vaisman manifolds (reminder)}

\[\begin{tikzcd}
	M \arrow[r,hook]&  \mathbb{C}P^{1}\\
	C_0(M)\arrow[u,"\mathbb{C}^*\text{-fibered} "]&  \mathbb{C}^{n+1}\setminus\{0\}\arrow[u,"\mathbb{C}^0"]\arrow[d,hook]\\
	C(M)\setminus \arrow[ u]\arrow[r,hook]&\mathbb{C}
\end{tikzcd}\]


\subsection{LCK manifolds in terms of differential forms}

So what is Kähler?

\((M,I)\) complex manifolds, $ g$ an \(I\)-invariant Riemannian metric, "Hermitian metric", \(\omega(x,y):=g(Ix,y)\) Hermitian form; \(d \omega=0\) Kähler.

\begin{defn}\leavevmode
	\(\omega\) Hermitian form, \(\omega \in \Lambda^{1,1}_{\mathbb{R}}(M)\), \(\omega(x,Ix)>0\), \(\omega\) is \textit{\textbf{Locally conformally Kähler}} if \(d \omega= \omega \wedge \theta\), \(\theta\) closed 1-form. \(\theta\) is called the \textit{\textbf{Lee form}}.
\end{defn}

\begin{remark}\leavevmode
	The condition if \textbf{conformally invariant}: it is preserved if we replace \(\omega\) by a conformally equivalent form \(f \omega\) for some positive smooth function \(f>0\).Indeed,
	\[d(f \omega)=df \wedge \omega+f d \omega=df \wedge \omega +f \theta \wedge \omega=(df+f\theta)\wedge\omega.\]
\end{remark}

This makes us notice that a classical Hopf manifold \(\frac{\mathbb{C}^n\setminus\{0\}}{\left< \lambda \operatorname{Id}\right>}\) is LCK.

\subsection{Chern connection again}

There is a connection on a holomorphic bundle compatible with the metric that is called \textit{\textbf{Chern connection}}.

The point is that the curvature can be written locally as  \(d d^c\) of some function. And it can be global if you have a non-degenerate holomorphic section taking \(\partial \bar\partial \operatorname{log}|b|\). But it is \(d d^c\) of a function that's the point.

Now there is 

\begin{thing6}{Theorem 5.30}[\cite{verbi}]\leavevmode
The function that maps \(l=\psi:v \mapsto |v|^2\) along with some other stuff like the definition of the function \(q\) then there following expression is true:
\[d d^c l=-q(\theta_B)+\omega_\pi.\]
\end{thing6}

Which leads to

\begin{coro}\leavevmode
	Let \(L\) be a line bundle with negative curvature on a projective manifold. Then the form \(\frac{ d d^c\psi}{\psi}\) is \textbf{homothety invariant} and locally conformally Kähler on \(\operatorname{Tot}^0(L)\).
\end{coro}

\begin{remark}\leavevmode
	We have just shown that \textbf{Vaisman manifolds are LCK}.
\end{remark}

\subsection{Homotheties, monodromy and objective}

We want to give a definition of LCK in terms of a Kähler form on the universal covering. Also might involve local systems. \textbf{Under the alternative definition, LCK manifold is a quotient of a Kähler manifold by a free action of cocompact, discrete group acting by homotheties.}

\begin{claim}\leavevmode
	Any conformal map \(\varphi:(M,\omega) \to (M_1,\omega_1)\) of Kähler manifolds is a homothety.
\end{claim}

\subsection{Reminder on connections and curvature}

The point is that local systems are flat line bundles.

\begin{defn}\leavevmode
	A \textit{\textbf{local system}} on a manifold is a locally constant sheaves of vector spaces.
\end{defn}

\begin{thm}[Rieman-Surfaces lecture 20]\leavevmode
Fix a point \(x \in M\). The category of local systems is naturally equivalent to the category of representations of \(\pi_1(M,x)\).
\end{thm}

\begin{proof}\leavevmode
\begin{enumerate}[label=\textbf{Step \arabic*}]
\item From a locally constant sheaf \(\mathbb{V}\) we construct a vector bundle \(B:=\mathbb{V} \otimes_{\mathbb{R}_M}\mathbb{C}^\infty M\), where \(\mathbb{R}_M\) is the constant sheaf on \(M\). Define a connection \(\nabla\left(\sum_{i=1}^n f_iv_i\right) =\sum df_i \otimes v_i\); where \(v_1,\ldots,v_n\) is a basis in \(\mathbb{V}(U)\). {\color{5}We have constructed a functor from locally constant sheaves to flat vector bundles.}

\item The converse functor takes a flat bundle \((B,\nabla)\) on \(M\) goes to the sheaf of parallel sections \(\nabla b=0\); this sheaf is  locally constant because every vector can be locally extended to a parallel section uniquely (using Frobenius theorem; this is non-trivial).
\end{enumerate}
\end{proof}

\subsection{\(\chi\)-automorphic forms}

The following resembles the way we have define LCK form on a manifold; multipliying by a number something that comes from the universal cover (…?)

\begin{defn}\leavevmode
	Let \(\tilde{M} \xrightarrow{\pi}M\) be the universal covering of \(M\), and \(\xi:\pi_1(M) \to \mathbb{R}^{>0}\) a \textit{\textbf{character}}, which is just a group homomorphism. Consider the natural action of \(\pi_1(M)\) on \(\tilde{M}\). An \textit{\textbf{\(\xi\)-automorphic form}} on \(\tilde{M}\) is a differential form \(\eta \in \Lambda^{k}(M)\) which satisfies \(\gamma^* \eta=\xi(\gamma)\eta\) for any \(\gamma \in \pi_{1}(M)\).

	This makes sense because \(\pi_1(M)\) acts freely on \(\tilde{M}\) (and the quotient is \(M\)), so we can pullback \(\eta\) and it gives an other form on \(\Lambda^{k}(M)\).
\end{defn}

\begin{thing6}{Proposition 1}[What Lada had said!, this is Claim 3.28 \cite{verbi}]\leavevmode
Let \(L\) be a rank 1 local system on \(M\) {\color{3}associated to the representation \(\chi\) (so how is it assocated to \(\chi\)?)}. Then the space of \(\chi\)-automorphic \(k\)-forms on \(\tilde{M}\) is in natural correspondence with the space of sections of \(\Lambda^{k}(M)\otimes L\). Under this equivalence, the de Rham differential on \(\chi\)-automorphic forms corrasponds to the operator \(d_\nabla: \Lambda^{k}(M)\otimes L \to \Lambda^{k+1}(M)\otimes L\).
\end{thing6}

\begin{proof}\leavevmode
\begin{enumerate}[label=\textbf{Step \arabic*}]
\item Pullback the line bundle to the universal cover: \(\tilde{L}:= \pi^*L\), \(\pi:\tilde{M} \to M\). I think \(\tilde{L}\) is trivial: ``The bundle \(\tilde{L}\) is flat and has trivial monodromy, hence it is naturally trivialized by parallel sections".
\end{enumerate}
\end{proof}

\begin{remark}\leavevmode
	\(d_\nabla=d+\theta\)
\end{remark}

\section{Lecture 5: local systems and LCK manifolds}

\subsection{\(\chi\)-automorphic forms again}

\begin{upshot}\leavevmode
	The point is that \(L\)-valued differential forms on \(M\) are in correspondence with \(\chi_L\)-automorphic differential forms \textit{on \(\tilde{M}\)}.
\end{upshot}

\begin{thing6}{Proposition 1}\leavevmode
\((L,\nabla)\) a real flat orientes line bundle. Identify with a local system: associated to \(\chi\) fix a trivializatino of \(L\). Then sections of \(L \otimes \Lambda^{1}(M)\) are in bijection with \(\chi\)-automorphic forms on \(\tilde{M}\) via
\begin{align*}
	\sigma: \Lambda^{\bullet}(M)\otimes L &\longrightarrow \Lambda^{\bullet}(\tilde{M}) 
\end{align*}
\[\sigma(d_\nabla \eta)=d\sigma(\eta).\]
\end{thing6}

\begin{proof}[Very incomplete proof]\leavevmode
\begin{enumerate}[label=\textbf{Step \arabic*}]
\item \(u_1\) a nowhere-vanishing section of \(L\), and \(\theta\) a 1-form such that \(\nabla u_1=u_1 \otimes \theta\).

	\begin{thing7}{Extra}\leavevmode
	How to produce the antiderivative of an exact one form: we integrate from \(x\) to \(y\).
	\end{thing7}
\end{enumerate}
\end{proof}

\subsection{Lichnerowicz cohomology}

Look for  \textbf{Definition 2.51} for definition of \(d_\nabla\), the \textit{\textbf{\(B\)-valued de Rham differential}} of the complex \(\Lambda^{i}(M) \otimes B \longrightarrow \Lambda^{i+1}(M) \otimes B\) given by \(d_\nabla(\eta \otimes b):=d \eta \otimes b +(-1)^{\tilde{\eta}-1}\eta \wedge \nabla b\) for the (real I think) flat bundle \((B,\nabla)\).

\begin{defn}\leavevmode
	Let \(\theta\) be a closed 1-form on a manifold, and \(d_\theta(\alpha):= d \alpha + \theta \wedge \alpha\) be the corresponding differential on \(\Lambda^{*}(M)\). Its cohomology are called \textit{\textbf{Morse-Novikov cohomology}}, or  \textit{\textbf{Lichnerowicz cohomology}}, denoted \(H^{^*}_\theta(M)\).
\end{defn}

\begin{thm}\leavevmode
	Lichnerowitz cohomology of a manifold is equal to the cohomology with coefficientes in a local system defined by \((L,\nabla)\).
\end{thm}

\begin{proof}\leavevmode
Short.
\end{proof}

\subsection{Definition of LCK manifolds in terms of an  \(L\)-calued Kähler form}

\begin{defn}\leavevmode
	Let \((L,\nabla)\) be an oriented real line bundle with flat connection on a complex manifold \(M\), and \(\omega \in L \otimes \Lambda^{ 1,1}(M)\) a \((1,1)\)-form with values in \(L\). We say that \(\omega\) is an \textit{\textbf{\(L\)-valued Kähler form}} if \(\omega(x, Ix) \in L\) is (strictly) positive for any non-zero tangent vector, and \(d_\nabla \omega =0\).
\end{defn}

\begin{thing6}{Superremark}\leavevmode
If we use a trivialization to identify \(L\) and \(C^\infty M\), \(\omega\) becomes a \((1,1)\)-form and \(d_\nabla\) becomes \(d_\theta\), giving \(d_\nabla(\alpha)= d\alpha + \theta \wedge \alpha\). Therefore, \textbf{\(L\)-valued Kähler form on a manifold is the same as an LCK-form}.
\end{thing6}

\subsection{Definition of LCK manifolds in terms of deck transform}

\begin{thing6}{Another definition}\leavevmode
An \textit{\textbf{LCK manifold}} is a complex manifold \(M\), \(\dim_\mathbb{C} M \geq 2\) such that its universal cover \(\tilde{M}\) is equipped with a Kähler form \(\tilde{\omega}\), and the deck transform acts on \(\tilde{M}\) by Kähler homotheties.
\end{thing6}

\begin{thm}\leavevmode
	These two definitions are equivalent.
\end{thm}

\section{Class 6: Vaisman theorem}

\subsection{LCK (reminder)}

\begin{defn}\leavevmode
A complex Hermitian manifold of dimension >1 \((M,I,g,\omega)\) is called \textit{\textbf{locally conformally Kähler}} if there is a closed form \(\theta\) such that \(d \omega =\theta \wedge \omega\). \(\theta\) is the \textit{\textbf{Lee form}} and its cohomology class is the \textit{\textbf{Lee class}}.
\end{defn}

\subsubsection*{The Deal with \(\omega\)}

\paragraph{Fundamental Form:}
\begin{itemize}
    \item The 2-form \(\omega\) is the \textbf{fundamental 2-form} associated with the Hermitian metric \(g\) and the complex structure \(J\), defined by:
    \[
    \omega(X, Y) = g(JX, Y).
    \]
    \item This \(\omega\) is not automatically a \textbf{symplectic form}, because it may fail to be \textbf{closed} (\(d\omega \neq 0\)).
\end{itemize}

\paragraph{Kähler Condition:}
\begin{itemize}
    \item When \(d\omega = 0\), \(\omega\) becomes a symplectic form, and the manifold \(M\) is Kähler. This means \(M\) is simultaneously a symplectic, complex, and Riemannian manifold with a harmonious interaction between these structures.
\end{itemize}

\paragraph{Locally Conformally Kähler:}
\begin{itemize}
    \item In LCK geometry, \(\omega\) doesn’t satisfy \(d\omega = 0\) globally. Instead, it satisfies:
    \[
    d\omega = \theta \wedge \omega,
    \]
    where \(\theta\) is the \textbf{Lee form} (a closed 1-form). This deviation from \(d\omega = 0\) characterizes LCK manifolds.
\end{itemize}

\paragraph{Why Locally Conformal?}
\begin{itemize}
    \item Locally, there exists a function \(f\) such that rescaling the metric by \(e^{-f}\) makes \(\omega\) closed:
    \[
    e^{-f}\omega \text{ is Kähler.}
    \]
    \item This means LCK manifolds are "almost" Kähler but need a conformal adjustment locally.
\end{itemize}

\subsubsection{Some more notes on complex geometry}

That the Kähler form is the differential of a plurisubharmonic function \(\psi\). that is \(\omega=d d^c \psi=\sqrt{-1} \partial \partial \psi\).

And it is a \((1,1)\)-form.

Any positive \((1,1)\) form looks like this:  \(\sum \alpha I x_i \wedge x_i\) for some positive functions \(\alpha_i \geq 0\).

How to prove adjunction formula, that the canonical bundle of a submanifold is the normal bundle of the submanifold tensor product the canonical bundle of the ambient manifold restricted to the submanifold, \(K_M=N_M \otimes K_X |_{M}\): contract a form of the ambient space with a normal section!

\subsection{Vaisman theorem}

Suppose you have a LCK manifold. Suppose you want to replace \(\theta\) by \(\theta'=\theta + df\).
\begin{align*}
d(e^f \omega )&= e^f( d \omega +df \wedge \omega)\\
&=e^f((\theta + df) \wedge \omega),\qquad \text{\(\omega\) is LCK} \\
&=e^f(\theta'\wedge\omega,\qquad \text{\(\theta\) and \(\theta'\) cohomologous} \\
\implies  d \omega'&=(\theta+ df) \wedge \omega'\\
\implies d \omega'&=\theta' \wedge \omega'.
\end{align*}
And the converse is also true:

\begin{quotation}
	conformally equivalent LCK metric give rise to homologous Lee forms, and any closed 1 form cohomologous to the Lee form is a Lee form conformally equivalent LCK metric.
\end{quotation}

\begin{thm}[Vaisman]\leavevmode
	\(M\) LCK, if \([\theta]=0\) (=\(\theta\) is not extact) then \(M\) is not of Kähler type. 
\end{thm}

\begin{remark}\leavevmode
\((M,I)\) Kähler compact: for \(\alpha \in \Lambda^{1}(M,\mathbb{C})\) there exists unique representatives \(\alpha=\alpha^{1,0}+\alpha^{0,1}\) that are closed, i.e.  \(d \alpha^{1,0}=d \alpha^{0,1}=0\).
\end{remark}

\begin{proof}[Proof of Vaisman theorem]\leavevmode
Take a form \(\theta\) and multiply it by its complex conjugate \(I \theta\). I think here we took local coordinates: \(\theta=x_1\) and \(I\theta=y_1\). So \(\omega = \sum x_i \wedge y_i\). And then here's what I don't understand:
\(x_1 \wedge y_1 \left(\sum_{i=1}^n x_i \wedge y_i\right)^{n-1}=(n-1)! x_1 \wedge y_1 \wedge \prod_{i=2}^4 x_i \wedge y_i\). And that's positive!

And that gives the contradiction that
\[0= \int d d^c (\omega^{n-1})>0\]
because any exact form has intergral zero by Stokes.
\end{proof}

\begin{defn}\leavevmode
\textit{\textbf{Vaismann manifold}} is \((M,I,\omega)\) complex hermitian, \(d\omega =\theta \wedge \omega\), \(\nabla\theta=0\), \(\nabla\) Levi-Civita connection of \(g=(\omega(x,Iy)\).

{\color{7}(I think here we may be considering the musical dual of \(\theta\) to take the covariant derivative.)}
\end{defn}

\begin{thing6}{Equivalent definition}\leavevmode
\(G \) complex Lie group acting on an LCK manifold conformally and holomorphically, then  \((M,I)\) is Vaismann.
\end{thing6}

\begin{remark}\leavevmode
The theorem that both definitions are equivalent is hard to prove.
\end{remark}

\subsection{Vaisman examples}
\begin{thm}\leavevmode
Diagonal Hopf is Vaisman.
\end{thm}

Diagonal Hopf is then the \(A\) in \(\mathbb{C}^n/\left<A\right>\) is diagonalizable.

I think

\begin{thm}\leavevmode
Z Hopf. \(Z\) is Vaisman iff is \(Z\) is diagonal.
\end{thm}

\subsection{The fundamental foliation}

\begin{defn}\leavevmode
\(M\) Vaisman manifold, \(\theta ^\sharp\) its Lee fild,, and \(\Sigma\) a 2-dimensional real foliation generated by \(\theta ^\sharp, I\theta ^\sharp\). It is called \textit{\textbf{the fundamental foliation}} of \(M\).
\end{defn}

\begin{question}\leavevmode
So at 
\end{question}

\begin{thm}\leavevmode
\(M\) Vaisman, \(\Sigma\) its canonical foliation.

\begin{enumerate}
\item \(\Sigma\) is independent from the metric (it's canonical).
\item There exists a 2-form \(\omega \in \Lambda^{1,1}(M)\) which is semipositive on every transversal to \(\Sigma\), \(\omega_0|_{\Sigma}=0\), exact.

{\color{6}It's a contact structure, right? On slides: 2. There exists a positive, exact \((1,1)\)-form \(\omega_0\) with \(\sum= \ker \omega_0\)}
	\begin{remark}\leavevmode
	This 2-form is easy to see in these examples: the pullback of Fubini-Study metric: its pullback is exact! Remember that the pullback of any bundle to the \(\operatorname{Tot}^0(M)\) is trivial.
	\end{remark}
\item  \(Z \subset M\) tangent to \(\Sigma\),
\item \(Z \subset M\) is Vaisman.
\end{enumerate}
\end{thm}

\section{Lecture 7: elliptic operators of order 2}

\begin{enumerate}
\item "The ring of symbols"
	\begin{thm}\leavevmode
		\[\bigoplus_{k}\frac{\operatorname{Dif}^k(M)}{\operatorname{Dif}^{k-1}(M)}=\bigoplus_{k}\operatorname{Sym}^{k-m}(TM)\]
	\end{thm}
	So on the lefthandside we have the graded algebra induced by the filtration \(\operatorname{Dif}^0 \subset \operatorname{Dif}^1 \subset \ldots\) of differential operators of different orders on \(M\).

	\begin{proof}\leavevmode
	We give a pairing
	\[\frac{\operatorname{Dif}^k}{\operatorname{Dif}^{k-1}}\otimes \frac{\mathfrak{m}^k}{\mathfrak{m}^{k-1}}\]
and we recall from Hartshorne that \(\frac{\mathfrak{m}^k}{\mathfrak{m}^{k+1}}=\operatorname{Sym}^k(T^*M)\).
	\end{proof}

	\item Definition: \textit{\textbf{symbol}} of the differential operator \(D \in \operatorname{Dif}^k(M)\) is its image in \(\operatorname{Sym}^k(TM)\)
	\item Remark. A differential operator gives us a polynomial function on the cotangent bundle because \(\operatorname{Sym}^k(T^*M)=\) order \(k\) homogeneous polynomial functions on \(T ^* M\).
	\item Definition: \(D \in \operatorname{Dif}^k\) is \textit{\textbf{elliptic}} if \(\sigma(D)\) is positive or negative everywhere on \(T^* M\setminus\{0\}\).
	\item Remark. The symbol of an elliptic operator of second order is positive definite or negative definite. We assume is positive definite. 
	\item 
		\begin{thing6}{Strong maximum principle (version with boundary)}[Hopf]\leavevmode
		\(M\) manifold with boundary. \(D\) elliptic of second order. $f \in C^\infty(M)$  \(D(f) \geq  0\). Then all local maxima of $f$ are on \(\partial M\) or $f$ is constant.
		\end{thing6}
		\begin{proof}[Proof assuming \(D(f)>0\)]\leavevmode
		Because the Hessian is negative semidefinite
		\end{proof}
	\item Weak maximum principle. I think here /cha
		\end{enumerate}

\section{Lecture 8: adjoint operators in Hodge theory}

\subsection{Adjoint connection (reminder)}

First recall that a connection on a vector bundle \(B\) induces a connection on the dual bundle: if \(\nabla: B \to B \otimes \Lambda^{1}(M)\), exists a unique connection \(\nabla^* : B ^*\to  B ^*  \otimes \Lambda^{1}(M)\) satisfying \(d \left< b , \beta\right>=\left<\nabla b, \beta\right>+ \left< b, \nabla ^* \beta\right>\).

The connection \(\nabla ^* \) is called \textit{\textbf{adjoint connection}} to \( \nabla\). The connection \(\nabla = \nabla ^*\) happens precisely when \(\nabla\) preserves the metric tension, consider a section of \( B ^* \otimes B ^* \) and in this case \(\nabla\) is called an \textit{\textbf{orthogonal connection}}.

\subsection{Adjoint connection and \(L^2\)-product}

\begin{upshot}\leavevmode
A scalar product on the space of sections of a vector bundle \(B\). Because we want to define adjoint differential operators on the infinite-dimensional space of sections of the bundle. You multiply sections pointwise, you obtain a function, you integrate that function, you get a number.
\end{upshot}

\(M\) Riemannian manifold, \(b, b'\) sections of \(B\).  \((,)\) scalar product on  \(B\), \((b,b,)_{L^2}=\int(b,b')\operatorname{Vol}\).

\begin{lemma}[Integration by parts]\leavevmode

\end{lemma}
\begin{proof}\leavevmode
The key observation is that
\[\int\mathsf{Lie}_X(\left<b,b'\right>)\operatorname{Vol}=0\]
because \(\mathsf{Lie}_X(\eta)=di_X\eta + \cancelto{0}{i_Xd\eta}\) so \(\mathsf{Lie}_X(\eta)\) is exact.
\end{proof}

\subsection{Adjoint operators}

\begin{defn}\leavevmode
\(A: F \to G\) linear map on vector spaces with scalar products. \(A: G \to G\) is \textit{\textbf{dual}} if \((Ax,y)=(x,A^*y)\). \(A^*y\) is a vector such that \(\left<A^*y,x\right>=\left<Ax,y\right>\).
\end{defn}

Existence is obvious and uniqueness (…)

\begin{claim}\leavevmode
A diferential operator on vector bundles with scalar products \(D : B_1 \to B_2\), then its adjoint \(D^*:B_2 \to B_1\) is also a differential operator of the same order.
\end{claim}

\begin{remark}\leavevmode
Most of you know that \(d^*=\pm  * d *\) which is a composition of linear operators, it's the Hodge star.
\end{remark}

\begin{proof}[Proof of claim]\leavevmode
\(C^\infty M\)-linear. Then we can take the dual point by point (dual exists because its finite dimensional), and it works because it's linear and it's a vector bundle. Then we claim that all first order differential operators are combinations of a fixed connection \(\nabla_X\) (connections are differential operators). Then somehow we have shown that actually \(\nabla_X \) coincides by \(\nabla_X^*\) (integrating by parts).
\end{proof}

\subsection{Laplacian on differential forms}

Start with a Riemannian manifold, say, compact. You can do it with non-compact: if you take care about having things with compact support, and it will work, but we don't want to do it because it takes some extra effort and we won't need it.

Then the sections of \(\Lambda^{*}(M)\) we define scalar product ver naturally:
 \[(\eta,\eta')_{L^2}=\int(\eta,\eta')\operatorname{Vol}_g\]
 now
 \[d^* = \text{dual to $d$} \]
 Also, but this is not related to our course since we did not define Hodge star operator, \(d^*=\pm  * d *\).
 
\begin{defn}\leavevmode
Laplacian is \(\Delta=dd^*+ d^*d\)
\end{defn}

\begin{remark}\leavevmode
It's self dual (self-adjoint) because \(*\) is self dual . And it's positive-definite 
\end{remark}

\section{Lecture 9: Atiyah-Singer index theorem}

Very famous theorem. It's topology. Application of analysis for topology.

\subsection{Fredholm operators}

\begin{defn}\leavevmode
A continuous operator \(F: H_1 \to H_2\) of Hilbert spaces is called \textit{\textbf{Fredholm}} if \textit{its image is closed} ({\color{6}that's new I think}) and its kernel and cokernel are finite dimensional. 
\end{defn}

Next is a condition of invertibility of Fredholm maps. Uses Banach-Schouder theorem. We take the kernel of the map and quotient its domain (we get injectivity); and the restrict the codomain to the image. But we do have to use that BS theorem.

\begin{claim}\leavevmode
\(F: H_1 \to H_2\) is Fredholm iff there is a map \(G:H_2 \to H_1\) such tat \(\operatorname{Id}-FG\) and \(\operatorname{Id}-GF\) have finite rank.
\end{claim}

\begin{defn}\leavevmode
\textit{\textbf{Index}} of \(F\) is \(\dim \ker F-\dim \operatorname{coker}F\).
\end{defn}

This one is also in \cite{brezis} (though not proved there):

\begin{prop}\leavevmode
The class of Fredholm operators is an open subset of \(\mathcal{L}(E,F)\) (with the norm topology, so the norm of an operator is the supremum of its values on the unit sphere).
\end{prop}

So that makes the domain of the index function a reasonable space. Then: \cite{brezis}: the index map \(A \mapsto  \operatorname{ind}A\)is continuous. But not only that:

\begin{thm}\leavevmode
The index function is locally constant.
\end{thm}

So, apparently obviously, the open set of domain of Fredholm operators is not connected, so for example, I think the shift function
So, apparently obviously, the open set of domain of Fredholm operators is not connected, so for example, I think the shift function.

\subsection{Sobolev norm}

\begin{defn}\leavevmode
\(B\) vector bundle with metric over a Riemannian manifold. Define \(L^2_p\) metric for asection \(b \in B\) 
\[|b|^2_p=|b|_{L^2_p}=\sum_{i=0}^p |\nabla^i b|\]
Of course in the non-compact case we must take things with compact support.
\end{defn}

\begin{remark}\leavevmode
This norm is not complete, must take closure.
\end{remark}

\begin{defn}\leavevmode
\(L^2_p\)\textit{\textbf{-topology}} on \(C^\infty M\) is topology defined by \(L^2_p\)-norm.
\end{defn}

\begin{remark}[should be simple]\leavevmode
If \(D\) is a differential operator (on a compact mfld for simplicity) of order \(k\). Then \(D : C^\infty  M,L^2_p \to C^\infty M,L^2_{p-k}\) is (continuous?)

If you have a bound on \(k\) derivatives and you take  more derivatives…
\end{remark}

\subsection{Elliptic operators}

\begin{defn}\leavevmode
\(D: B \to B\) order \(k\) differential operator, then \( \operatorname{symb}(D) \in \operatorname{Sym}^k(T)\otimes \operatorname{End}(B)\) \(k\) (this is agrued using a matrix of symbols) =degree \(k\) polynomial functions on \(T^*M\) with values in \(\operatorname{End}(B)\).

\(D\) is \textit{\textbf{elliptic}} if (after interpreting its symbol as a polynomial function) \(\operatorname{symb}(D)(v)\) is invertible fora ll \(0 \neq  v \in T^*M \).
\end{defn}

\begin{thm}[Elliptic operator is Fredholm (I didn't find it in \cite{brezis})]\leavevmode
\(B_1,B_2\) vector bundles, \(D: B_1 \to B_2\) elliptic of order $p$. Then \(D(B_1,L^2_{k+p}\to (B_2,L^2_k)\) is Fredholm.
\end{thm}

Super hard to prove but super important and basic.

\begin{defn}\leavevmode
\(D: B_1 \to BV_2\) elliptic. Its \textit{\textbf{index}} is the index of the map \(D:(B_1,L^2_p \to (B_2,L^2)\)\end{defn}

\begin{coro}\leavevmode
Let \(D_t\) be a continuous family of elliptic operators. The map \(t \mapsto \operatorname{ind}(D_t)\) is constant. so that if \(D_t\) are elliptic they remain elliptic.
\end{coro}

\begin{proof}\leavevmode
Because we have seen that index is locally constant.
\end{proof}

\section{Index theorem for elliptic operators on \(C^\infty M\)}

\begin{thm}\leavevmode
All elliptic operators on \(C^\infty M\) have index 0.
\end{thm}

\begin{proof}\leavevmode
\begin{enumerate}[label=\textbf{Step \arabic*}]
\item \(\operatorname{symb}D\) is a homogeneous function. (by properties of the symbol) we conclude that \(\operatorname{deg}D\) is even.
\end{enumerate}
\end{proof}

\bibliography{bib.bib}
\end{document}
